\documentclass[12pt]{article}
\usepackage[margin=2cm]{geometry}
\usepackage{hyperref}
\usepackage{natbib}
\usepackage{setspace}
\setstretch{1.5}


\begin{document}
\begin{titlepage}
    \centering
    \vspace*{1.5cm}
    \Huge
    \textbf{Accelerating Uganda's Vision 2040 Through Affordable and Clean Energy}
    
    \vspace{1.5cm}
    \LARGE
    Aligning with Global Sustainable Development Goals
    
    \vspace{2cm}
    \Large
    \textbf{Research Project}

    \vspace{2.5cm}
    \Large
    \textbf{Prepared By:}
    
    \vspace{0.5cm}
    \large
    Ezamati Ronald A\\
    Mucunguzi Godfrey\\
    \date{\today}

    \vspace*{1cm}
\end{titlepage}

Uganda's Vision 2040 aims to transform the nation into a competitive upper-middle-income economy by 2040. A central pillar of this vision is the development of affordable, reliable, and sustainable energy, aligning with Global Sustainable Development Goal (GSDG) 7: \textit{Affordable and Clean Energy}. Energy access is essential for industrialization, socio-economic growth, and improved quality of life. However, Uganda faces challenges, including low rural electrification rates, overreliance on biomass, and limited diversification of renewable energy sources.

To address these issues, Vision 2040 emphasizes scaling up electricity access to 80\% and increasing per capita consumption to 3,668 kWh by 2040. The strategy includes harnessing renewable energy sources such as hydropower, solar, and geothermal energy. Major projects like the Karuma Hydropower Plant (600 MW) have been commissioned to expand electricity generation capacity. 

Despite progress, challenges remain. Overdependence on hydropower makes the energy sector vulnerable to climate variability, while the high cost of renewable energy technologies limits adoption. Addressing these barriers requires policy interventions, investment in decentralized energy systems, and enhanced public-private partnerships to accelerate the adoption of clean energy.

This research aims to evaluate Uganda's energy strategies, identify challenges and recommend actionable solutions to achieve GSDG 7 targets. By aligning national energy initiatives with global best practices, Uganda can ensure inclusive and sustainable development.

\bigskip

\noindent \textbf{References}
\begin{itemize}
    \item Global Sustainable Development Goal 7: Affordable and Clean Energy. Retrieved from \url{https://sdgs.un.org/goals/goal7}
    \item Uganda Vision 2040. Retrieved from \url{https://www.greenpolicyplatform.org/sites/default/files/downloads/policy-database/UGANDA%29%20Vision%202040.pdf}
\end{itemize}

\end{document}
