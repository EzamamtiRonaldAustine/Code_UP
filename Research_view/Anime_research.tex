\documentclass[12pt]{article}
\usepackage{geometry}
\geometry{a4paper, margin=1.5cm}
\usepackage{natbib}
\usepackage{setspace}
\usepackage{times}
\setstretch{1.5}

\title{\Large \textbf{The Power of Anime: Artistic Power, Social Consciousness, and Cultural Impact}}
\author{Natalie Ortez-Arevalo}
\date{December 16, 2022}

\begin{document}

\maketitle

\section*{Article Information}
\textbf{Published in:} Master's Projects and Capstones, University of San Francisco 

\section*{Article Summary}
The article explores the impact of anime on Japanese culture and society, arguing that anime serves as a medium for artistic storytelling, social commentary, and cultural expression. The research highlights how anime reflects historical and contemporary issues, such as wartime atrocities, environmental concerns, gender and sexuality discussions, and mental health awareness.

The author examines anime's paradoxical nature, acknowledging both its positive and negative influences. While anime can spread propaganda and objectify characters, it predominantly acts as a positive force, raising awareness and fostering social discussions.

Using various sources and case studies, \cite{ortez2022power} discusses notable anime productions like \textit{Akira, Grave of the Fireflies, and Nausicaä of the Valley of the Wind}, demonstrating how these works tackle critical societal themes. The study also examines anime's economic significance and its role in shaping both domestic and international perspectives on Japanese culture.

\bibliographystyle{apalike}
\bibliography{Reference}

\end{document}
