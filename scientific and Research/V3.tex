\documentclass{IEEEtran}

% Required packages
\usepackage{cite}
\usepackage{amsmath,amssymb,amsfonts}
\usepackage{graphicx}
\usepackage{textcomp}
\usepackage{xcolor}
\usepackage[pdfborder={0 0 0}]{hyperref}
\usepackage{url}
\usepackage{hyperref}
\hypersetup{
  colorlinks=true,
  linkcolor=black,
  citecolor=black,
  filecolor=black,
  urlcolor=black
}
\usepackage{multirow}
\usepackage{array}
\usepackage{balance}
\usepackage{fancyhdr}

% IEEE Access specific settings
\def\BibTeX{{\rm B\kern-.05em{\sc i\kern-.025em b}\kern-.08em
    T\kern-.1667em\lower.7ex\hbox{E}\kern-.125emX}}

% Define IEEE blue color for section headers
\definecolor{IEEEblue}{RGB}{0, 102, 204}

% Apply color to section and subsection headers
\usepackage{titlesec}
\titleformat{\section}[block]{\large\bfseries\color{IEEEblue}}{\thesection.}{1em}{}
\titleformat{\subsection}[block]{\normalsize\bfseries\color{IEEEblue}}{\thesubsection}{1em}{}

% Professional header with IEEE Access left and title right
\pagestyle{fancy}
\fancyhf{}
\fancyhead[L]{\small\textit{IEEE Access}}
\fancyhead[R]{\small\textit{IoT-Based Water Quality Monitoring for Sustainable Aquaculture}}
\fancyfoot[C]{\thepage}
\renewcommand{\headrulewidth}{0.5pt}
\renewcommand{\footrulewidth}{0pt}

\begin{document}

% Title
\title{IoT-Based Water Quality Monitoring for Sustainable Aquaculture: A Systematic Review}

% Authors
\author{
    \IEEEauthorblockN{Ezamamti Ronald Austine, Kisa Emmanuel, and Tendo Calvin}
    \IEEEauthorblockA{ \\
        \textit{Department of Computing and Technology}\\
        \textit{Uganda Christian University}\\
        Mukono, Uganda\\
        Email: \url{B24252@Students.ucu.ac.ug}
    }
}

% Generate the title
\maketitle

% Keep title page clean
\thispagestyle{plain}

% Funding acknowledgment footnote
% \blfootnote{Manuscript received January XX, 2025; revised January XX, 2025. This work was supported by Uganda Christian University through access to research facilities and library resources.}

% Abstract
\begin{abstract}
Water quality management critically impacts aquaculture productivity in developing regions. This systematic review examines IoT-based monitoring systems for fish pond management following PRISMA 2020 guidelines. Database SCOPUS was systematically searched, yielding 89 records. After duplicate removal and screening, 12 peer-reviewed studies (2020--2025) met inclusion criteria. Results indicate IoT systems integrating dissolved oxygen, pH, temperature, and turbidity sensors significantly improve aquaculture outcomes. Key technologies comprise microcontrollers (ESP32, Arduino, STM32), wireless protocols (NB-IoT, GSM), and machine learning algorithms. Quantified benefits include 33.3\% survival rate improvement, 99\% correlation for parameter prediction, and economically viable solar operation at \$0.61/kWh. Quality assessment rated 8 studies as high quality, 3 as medium, and 1 as low. Critical gaps exist in affordable system implementation for smallholder farmers in resource-constrained environments. Design recommendations emphasize phased deployment, GSM communication, solar power, and community support networks targeting <\$150 systems. This review establishes evidence-based guidelines for contextually appropriate IoT aquaculture solutions in developing regions.
\end{abstract}

% Keywords
\begin{IEEEkeywords}
Aquaculture, dissolved oxygen, fish pond management, Internet of Things, machine learning, systematic literature review, water quality monitoring.
\end{IEEEkeywords}

% INTRODUCTION
\section{Introduction}
\IEEEPARstart{A}{quaculture} contributes critically to global food security, providing essential protein to billions worldwide \cite{Adeleke2020}. However, production challenges persist, particularly in developing regions where water quality management inadequacies cause substantial losses.

\subsection{Problem Statement and Context}
Uganda exemplifies these challenges. Between 2023 and 2024, fish production declined 27.8\% and export revenues dropped 21.9\%, with over 100 tonnes of fish reportedly lost in 2021 due to low dissolved oxygen and related water quality issues around Lake Victoria. Traditional manual monitoring proves time-consuming, expensive, and fails to provide real-time feedback, resulting in delayed responses to critical water quality changes.

Research demonstrates water quality parameters---temperature, pH, and ammonia---significantly affect fish weight and size \cite{Tumwesigye2022}. In Uganda's Ibanda District, only four of eight examined parameters remained within acceptable ranges; ammonia, temperature, pH, and iron exceeded recommended levels, directly impacting productivity \cite{Tumwesigye2022}. Despite government interventions, smallholder farmers constituting the majority of aquaculture producers face persistent challenges maintaining optimal pond conditions.

\subsection{Technological Solution and Research Gap}
Internet of Things (IoT) technologies offer promising solutions through continuous real-time monitoring, automated control, and data-driven decision-making \cite{Manoj2022}. Global adoption of IoT aquaculture systems has increased \cite{Antony2020}, yet implementation in resource-constrained environments remains limited. Smallholder farmers require affordable, energy-efficient, locally maintainable solutions operating within infrastructural constraints: limited electricity access, intermittent internet connectivity, and restricted technical support.

Comparative reviews identify technology integration as a critical African aquaculture challenge \cite{Adeleke2020}. While IoT systems demonstrate success in controlled research environments, their adaptation for smallholder farmers in East Africa remains largely unexplored. This gap proves particularly critical given that small-scale aquaculture in Uganda's Lake Victoria basin shows significant production potential but lacks technological integration addressing water quality management.

\subsection{Research Objectives}
This systematic literature review addresses five research questions:
\begin{enumerate}
\item What IoT technologies and system architectures are employed in aquaculture water quality monitoring?
\item Which water quality parameters are most critical for fish pond management?
\item How effective are IoT-based systems in improving aquaculture productivity?
\item What challenges exist in implementing IoT aquaculture systems in developing regions?
\item What design considerations are necessary for creating affordable, sustainable IoT solutions for smallholder aquaculture?
\end{enumerate}

\subsection{Contributions}
This systematic literature review contributes by: \textbf{(1)} providing comprehensive synthesis of IoT-based water quality monitoring technologies validated through PRISMA 2020 methodology; \textbf{(2)} identifying quantifiable benefits and critical parameters through analysis of 12 peer-reviewed studies; \textbf{(3)} highlighting significant research gaps addressing resource-constrained environments and smallholder needs; \textbf{(4)} developing evidence-based design recommendations tailored for developing regions emphasizing affordability, maintainability, and cultural appropriateness; \textbf{(5)} establishing foundations for future research on contextually appropriate IoT aquaculture solutions in East Africa and similar regions.

% METHODOLOGY
\section{Methodology}
This systematic literature review follows PRISMA 2020 guidelines ensuring transparent, comprehensive reporting \cite{Manoj2022}.

\subsection{Search Strategy}
Systematic search was conducted in the SCOPUS database in January 2025. The search string combined IoT terminology, aquaculture contexts, and water quality parameters: \textit{("IoT" OR "Internet of Things" OR "wireless sensor" OR "smart system") AND ("aquaculture" OR "fish pond" OR "fish farm*") AND ("water quality" OR "monitoring" OR "dissolved oxygen" OR "pH" OR "temperature" OR "turbidity")}.

\textbf{Inclusion criteria} required: \textbf{(1)} peer-reviewed journal articles indexed in SCOPUS; \textbf{(2)} published between 2020 and 2025; \textbf{(3)} written in English; \textbf{(4)} focused on IoT-based water quality monitoring for fish pond aquaculture; \textbf{(5)} including empirical data, system design, or implementation details. 

\textbf{Exclusion criteria} eliminated: \textbf{(1)} review papers without original research contributions; \textbf{(2)} studies focusing solely on marine or open water systems; \textbf{(3)} papers not involving IoT or embedded systems; \textbf{(4)} unavailable full texts; \textbf{(5)} duplicate publications; \textbf{(6)} studies exclusively on fish disease diagnosis without water quality monitoring.

\subsection{Study Selection Process}
Initial search yielded 89 records from SCOPUS. After removing 15 duplicates, 74 unique records underwent title and abstract screening. Papers not meeting inclusion criteria were excluded (n=42), leaving 32 for full-text assessment. 

Full texts were independently assessed by two reviewers; disagreements were resolved through discussion. Final exclusions included: not focused on pond aquaculture (n=12), no IoT implementation (n=5), full text unavailable (n=2), not indexed in SCOPUS (n=1). Final analysis included 12 papers meeting all criteria. The PRISMA flow diagram appears in Figure \ref{fig:prisma}.

\begin{figure}[!t]
\centering
\includegraphics[width=0.9\columnwidth]{prisma_diagram.pdf}
\caption{PRISMA 2020 flow diagram showing systematic study selection from database search to final inclusion of 12 studies.}
\label{fig:prisma}
\end{figure}

\subsection{Data Extraction}
A standardized extraction form captured: study characteristics (author, year, location, study type), system architecture details (microcontrollers, sensors, communication protocols), monitored parameters (DO, pH, temperature, turbidity, ammonia), control mechanisms (aeration, water exchange), power solutions (solar, grid, battery), outcomes (survival rates, accuracy, cost), challenges (technical, contextual), and cost considerations.

\subsection{Quality Assessment}
Quality assessment evaluated five criteria per study: \textbf{(1)} methodology clarity and replicability; \textbf{(2)} technical specification completeness; \textbf{(3)} system validation evidence; \textbf{(4)} limitation reporting; \textbf{(5)} real-world applicability. Studies were rated as high quality (meeting 4--5 criteria), medium quality (2--3 criteria), or low quality (0--1 criteria). This assessment informed synthesis and evidence strength identification.

\subsection{Quality Assessment Results}
Of 12 included studies, 8 rated as high quality \cite{Huan2020, Tsai2022, Jamroen2023, Baena-Navarro2025, Li2022, Ren2020, Shete2024, Mramba2023}, 3 as medium quality \cite{Antony2020, Manoj2022, Tumwesigye2022}, and 1 as low quality \cite{Adeleke2020}. 

High-quality studies provided comprehensive technical specifications, validation data, and acknowledged limitations. Medium-quality studies offered valuable insights but lacked complete implementation details or controlled environment validation. The low-quality study provided important African aquaculture context but minimal technical IoT implementation details. All studies contributed meaningfully to answering research questions despite quality variations.

% RESULTS
\section{Results}

\subsection{Overview of Included Studies}
Table \ref{tab:studies} summarizes the 12 included studies. Publications increased from 2022 onwards. Geographic distribution included Asia (n=7), Africa (n=3), South America (n=1), and global reviews (n=1). Study types comprised experimental implementations (n=7), prototype developments (n=3), field deployments (n=1), and comparative reviews (n=1).

\begin{table*}[!t]
\caption{Summary of Included Studies (N=12)}
\label{tab:studies}
\centering
\footnotesize
\renewcommand{\arraystretch}{1.3}
\begin{tabular}{|p{2.0cm}|c|p{2.5cm}|p{3.0cm}|p{4.8cm}|}
\hline
\textbf{Author, Year} & \textbf{Region} & \textbf{Parameters} & \textbf{Technologies} & \textbf{Main Findings} \\
\hline\hline
Huan et al. \cite{Huan2020} & China & DO, pH, Temp & STM32, NB-IoT & High accuracy: Temp ±0.12°C, DO ±0.55 mg/L, pH ±0.09 \\
\hline
Ren et al. \cite{Ren2020} & China & DO, Temp & Deep Belief Network & Superior DO prediction accuracy over traditional methods \\
\hline
Tsai et al. \cite{Tsai2022} & Taiwan & DO, pH, Temp & Fuzzy control, auto aeration & 33.3\% survival improvement in shrimp \\
\hline
Li et al. \cite{Li2022} & China & DO, pH, NH$_3$, N & SVM, IoT sensors & 99\% correlation for water quality prediction \\
\hline
Jamroen et al. \cite{Jamroen2023} & Thailand & DO, pH, Temp, Turb & 50Wp solar, NB-IoT & 100\% reliability, \$0.61/kWh energy cost \\
\hline
Mramba et al. \cite{Mramba2023} & Tanzania & DO, pH, Temp, Turb & Arduino, WSN & Non-linear DO-yield relationship established \\
\hline
Shete et al. \cite{Shete2024} & India & DO, pH, Temp & Arduino, IoT cloud & Reduced disease via parameter correlation \\
\hline
Baena-Navarro et al. \cite{Baena-Navarro2025} & Colombia & DO, pH, Temp & Random Forest, QAOA & >90\% survival, 6000+ interventions \\
\hline
Tumwesigye et al. \cite{Tumwesigye2022} & Uganda & Temp, pH, NH$_3$ & Survey methodology & Water parameters affect fish weight/size \\
\hline
Adeleke et al. \cite{Adeleke2020} & Africa & Various & Literature review & Technology integration is critical challenge \\
\hline
Antony et al. \cite{Antony2020} & Global & DO, pH, Temp & Review of IoT & Implementation barriers for smallholders \\
\hline
Manoj et al. \cite{Manoj2022} & Review & DO, pH, Temp & State-of-art review & Comprehensive technology assessment \\
\hline
\end{tabular}
\end{table*}

\subsection{Critical Water Quality Parameters}
\textbf{Dissolved Oxygen (DO):} The most critical parameter, monitored in all 12 studies. Optimal ranges: 5--8 mg/L for tilapia and catfish. Research confirms DO levels significantly impact fish survival and growth \cite{Tumwesigye2022}, demonstrating non-linear relationships with fish yield \cite{Mramba2023}.

\textbf{pH:} Monitored in 11 studies using electrochemical sensors. Optimal ranges: 6.5--8.5 for most species. Uganda-specific research indicates pH deviations adversely affect fish weight and size \cite{Tumwesigye2022}.

\textbf{Temperature:} Universal across all studies using DS18B20 or thermistor sensors. Optimal range for tropical species (e.g., Nile tilapia): 25--30°C. Control accuracy: ±0.12°C to ±0.5°C \cite{Huan2020}.

\textbf{Turbidity:} Monitored in 7 studies assessing water clarity and suspended solids, serving as water quality deterioration indicators \cite{Jamroen2023, Mramba2023}.

\textbf{Ammonia and Nitrogen:} Monitored in 5 studies, though sensor reliability and cost remain challenges. Research identifies ammonia as significantly affecting fish productivity \cite{Li2022, Tumwesigye2022}.

\subsection{IoT System Architectures and Technologies}
\textbf{Hardware platforms} include: STM32 microcontrollers for robust performance \cite{Huan2020}, Arduino-based systems for research contexts \cite{Shete2024, Mramba2023}, and ESP32 platforms for WiFi-enabled applications \cite{Antony2020}. 

\textbf{Communication technologies} feature: NB-IoT for long-range, low-power communication achieving 0.89\% packet loss rates \cite{Jamroen2023}; GSM-based communication appears most practical for rural deployments in developing regions with existing cellular infrastructure.

\subsection{Automation and Machine Learning Integration}
Ten studies implemented automated aeration triggered when DO levels dropped below thresholds (typically 4--5 mg/L). Tsai et al. \cite{Tsai2022} demonstrated automated aeration increased shrimp survival rates by 33.3\% compared to conventional systems. Five studies incorporated automated water exchange mechanisms activated when multiple parameters exceeded safe ranges.

Four studies employed machine learning for predictive management. Ren et al. \cite{Ren2020} developed Deep Belief Network models for DO prediction with superior accuracy. Li et al. \cite{Li2022} demonstrated Support Vector Machines achieving 99\% correlation coefficients for DO, pH, and nitrogen predictions. Baena-Navarro et al. \cite{Baena-Navarro2025} integrated Random Forest with Quantum Approximate Optimization Algorithm, achieving 50\% training time reduction while maintaining high accuracy (R²=0.999, RMSE=0.0998 mg/L for DO).

\subsection{Power Supply Solutions and System Effectiveness}
Jamroen et al. \cite{Jamroen2023} conducted comprehensive techno-economic optimization of standalone photovoltaic systems, identifying optimal configuration: 50 Wp PV capacity with 480 Wh battery storage, achieving 100\% reliability with levelized cost \$0.61/kWh. This provides valuable benchmarks for equatorial locations with consistent solar irradiance.

Quantitative benefits include: 33.3\% survival improvement \cite{Tsai2022}; reduced disease incidence through parameter correlation analysis \cite{Shete2024}; 90\% survival rates maintained through 6000+ corrective interventions \cite{Baena-Navarro2025}; high measurement accuracy---temperature ±0.12°C (0.15\% error), DO ±0.55 mg/L (2.48\% error), pH ±0.09 (0.21\% error) \cite{Huan2020}.

\subsection{Implementation Challenges}
\textbf{Technical challenges} include: sensor drift and fouling, calibration requirements, energy constraints in off-grid locations, and limited rural infrastructure \cite{Shete2024, Manoj2022}. 

\textbf{Context-specific challenges} for developing regions include: high initial sensor costs, limited local technical capacity, inadequate infrastructure (electricity, internet), need for culturally appropriate interfaces, ongoing calibration costs \cite{Antony2020}. East African studies emphasize farmer training in proper management practices as essential \cite{Mramba2023, Tumwesigye2022}.

% DISCUSSION
\section{Discussion}

\subsection{Synthesis of Key Findings}
This systematic review provides substantial evidence supporting IoT-based water quality monitoring in improving aquaculture outcomes. Unanimous focus on dissolved oxygen, pH, temperature, and turbidity confirms these as fundamental parameters requiring continuous monitoring. Research from Uganda and Tanzania specifically validates that temperature, pH, and ammonia significantly affect fish weight and size \cite{Tumwesigye2022, Mramba2023}, directly supporting real-time monitoring necessity in African contexts.

Automated aeration systems represent proven intervention strategies with documented 33.3\% survival improvements \cite{Tsai2022}, particularly relevant given fish losses attributed to low dissolved oxygen in East African water bodies. Machine learning integration demonstrates evolution from reactive to predictive pond management, though implementation requires historical data, computational resources, and validation across different conditions. Techno-economic analysis \cite{Jamroen2023} demonstrates solar-powered IoT systems are technically feasible and economically viable; equatorial locations offering consistent solar irradiance enable potentially more cost-effective implementations in tropical regions.

\subsection{Critical Research Gaps}
While 12 studies demonstrate global technical feasibility, only 3 address African contexts \cite{Tumwesigye2022, Adeleke2020, Mramba2023}. Critical gaps include: absence of cost-benefit analyses for smallholder farmers in developing regions; insufficient attention to local maintenance capabilities; limited research on culturally appropriate interfaces for varying literacy levels; minimal investigation of community-based support models. Comparative review \cite{Adeleke2020} highlights significant potential but identifies technological integration as a key challenge across African aquaculture systems.

High sensor costs remain a primary barrier for smallholder adoption. Research opportunities exist in: evaluating lower-cost sensor alternatives validated for specific contexts; developing calibration protocols using locally available reference solutions; investigating sensor sharing strategies across multiple small ponds; establishing local supply chains for maintenance and replacement.

None of the reviewed studies examined integration of IoT systems with traditional aquaculture knowledge. For successful adoption in developing regions, research should address: how IoT systems complement traditional observation methods; training approaches for farmers with limited technological background; user interface designs appropriate for local contexts; validation of system recommendations against farmer experiential knowledge.

\subsection{Implications for Resource-Constrained Environments}
Based on systematic review findings and identified gaps, we propose design principles for IoT-based fish pond monitoring suitable for smallholder aquaculture in developing regions. 

\textbf{Core components} should include: affordable microcontroller platform (considering cost and local availability); waterproof sensors (temperature, pH, DO, turbidity); GSM module for SMS alerts and periodic data transmission; solar PV system (40--60 Wp) with sealed battery storage (30+ Ah capacity); relay modules for automated pump control.

\textbf{Communication strategy} should leverage: GSM-based communication using existing cellular infrastructure; SMS alerts for critical events; periodic cloud data transmission when internet available; local data logging for continuity during connectivity loss.

\textbf{Implementation strategy} includes: phased deployment beginning with monitoring-only systems to build farmer trust and generate baseline data; implementing automated control after demonstrating monitoring value; enabling incremental feature additions as resources permit.

\textbf{Community support} should establish: local technical support networks; training programs appropriate for varying education levels; farmer cooperatives for shared maintenance resources.

\textbf{Cost optimization} should target: system cost below \$150 for basic monitoring configuration; prioritize sensors with longest calibration intervals; design for 3--5 year operational lifetime; provide clear cost-benefit projections based on mortality reduction.

\subsection{Limitations}
This review has several limitations. Focus on SCOPUS may have excluded relevant grey literature and regional technical reports. English language restriction potentially excluded valuable research in local languages. Temporal scope (2020--2025) may have missed earlier foundational work. Heterogeneity in study designs complicated direct comparisons. Few studies provided long-term operational data beyond pilot implementations; limited economic analysis existed across different regional contexts. Additionally, limited studies from African contexts restrict generalizability of findings to resource-constrained environments.

% CONCLUSION
\section{Conclusion}
This systematic literature review provides comprehensive evidence that IoT-based water quality monitoring systems significantly improve aquaculture outcomes through real-time monitoring, automated control, and data-driven decision-making. Analysis of 12 peer-reviewed studies confirms monitoring dissolved oxygen, pH, temperature, and turbidity enables effective pond management; automated interventions prevent critical water quality deterioration.

Key findings demonstrate quantifiable benefits: 33.3\% survival improvement through automated aeration; high measurement accuracy suitable for production environments; successful predictive management using machine learning. Solar-powered operation is technically feasible and economically viable, with optimized configurations achieving 100\% reliability at \$0.61/kWh.

However, significant research gaps exist in applying these technologies to resource-constrained environments. Only three studies addressed African contexts, with limited focus on smallholder farmer needs, cost optimization for developing regions, culturally appropriate interfaces, and integration with traditional practices. For regions where water quality parameters demonstrably affect fish productivity and small-scale aquaculture shows significant potential, localized IoT solutions remain underdeveloped.

Design recommendations emphasize: phased deployment strategies; community-based support networks; careful cost optimization targeting \$150 systems with 3--5 year lifespans. GSM communication leveraging existing cellular infrastructure appears most practical for rural deployments; solar power suits equatorial regions' consistent irradiance patterns.

Future research should prioritize: cost-benefit analyses for smallholder farmers; validation of lower-cost sensor alternatives; development of culturally appropriate training materials; long-term field studies in East African and similar resource-constrained contexts. As global aquaculture increasingly adopts IoT technologies, ensuring these benefits reach smallholder farmers in developing regions requires continued research attention and culturally sensitive design approaches tailored to local constraints and opportunities.

% Acknowledgment
\section*{Acknowledgment}
The authors acknowledge Uganda Christian University for supporting this research through provision of research facilities and library resources.

% Bibliography
\bibliographystyle{IEEEtran}
\bibliography{fish_pond_references}

% Author Biographies
\begin{IEEEbiography}[{\includegraphics[width=1in,height=1.25in,clip,keepaspectratio]{img/images.jpg}}]{Ezamamti Ronald Austine}
is currently pursuing the B.Sc. degree in Computer Science with the Department of Computing and Technology, Uganda Christian University, Mukono, Uganda. His research interests include Internet of Things applications in agriculture, embedded systems design, wireless sensor networks, and sustainable aquaculture technology solutions for resource-constrained environments in developing regions. He is particularly focused on developing affordable IoT systems for smallholder farmers in East Africa.

Mr. Austine has been involved in several research projects focusing on smart agriculture and environmental monitoring systems. (e-mail: \href{mailto:B24252@Students.ucu.ac.ug}{B24252@Students.ucu.ac.ug}; ORCID: \href{https://orcid.org/0009-0009-0005-2269}{0009-0009-0005-2269})
\end{IEEEbiography}

\begin{IEEEbiography}[{\includegraphics[width=1in,height=1.25in,clip,keepaspectratio]{img/images.jpg}}]{Kisa Emmanuel}
is currently pursuing the B.Sc. degree in Computer Science with the Department of Computing and Technology, Uganda Christian University, Mukono, Uganda. His research interests include wireless sensor networks, smart farming technologies, data analytics for agricultural applications, and IoT system integration for real-time monitoring solutions.

Mr. Emmanuel has contributed to research on sustainable technology solutions for rural communities and has experience in developing sensor-based monitoring systems. (e-mail: \href{mailto: B24259@Students.ucu.ac.ug}{ B24259@students.ucu.ac.ug}; ORCID: \href{https://orcid.org/0009-0003-3977-3676}{0009-0003-3977-3676})
\end{IEEEbiography}

\begin{IEEEbiography}[{\includegraphics[width=1in,height=1.25in,clip,keepaspectratio]{img/images.jpg}}]{Tendo Calvin}
is currently pursuing the B.Sc. degree in Computer Science with the Department of Computing and Technology, Uganda Christian University, Mukono, Uganda. His research interests include water quality sensing technologies, embedded systems programming, microcontroller applications, and automation systems for aquaculture management.

Mr. Calvin has worked on projects involving sensor calibration, data acquisition systems, and automated control mechanisms for environmental monitoring. (e-mail: \href{mailto:B24247@students.ucu.ac.ug}{B24247@Students.ucu.ac.ug}; ORCID: \href{https://orcid.org/0009-0007-8826-9619}{0009-0007-8826-9619})
\end{IEEEbiography}

\end{document}