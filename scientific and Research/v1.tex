\documentclass[conference]{IEEEtran}
\usepackage{cite}
\usepackage{amsmath,amssymb,amsfonts}
\usepackage{algorithmic}
\usepackage{graphicx}
\usepackage{textcomp}
\usepackage{xcolor}
\usepackage{url}
\usepackage{hyperref}

\begin{document}

\title{IoT-Based Water Quality Monitoring Systems for Sustainable Aquaculture: A Systematic Literature Review}

\author{\IEEEauthorblockN{Ezamamti Ronald Austine, Kisa Emmanuel, and Tendo Calvin}
\IEEEauthorblockA{\textit{Department of Computing and Technology, Faculty of Engineering, Design and Technology} \\
\textit{Uganda Christian University}\\
Mukono, Uganda \\
ezamautine@gmail.com, ekisa@ucu.ac.ug, ctendo@ucu.ac.ug}
}

\maketitle

\begin{abstract}
Aquaculture productivity is increasingly challenged by poor water quality management, particularly in developing regions where manual monitoring methods prevail. This systematic literature review examines IoT-based water quality monitoring systems for fish pond management, following PRISMA 2020 guidelines. We conducted a comprehensive search of SCOPUS and Web of Science databases, identifying 12 relevant peer-reviewed studies published between 2020 and 2025. Our analysis reveals that IoT-enabled systems integrating sensors for dissolved oxygen, pH, temperature, and turbidity significantly improve aquaculture outcomes through real-time monitoring and automated control mechanisms. Key technologies include microcontrollers such as ESP32, Arduino, and STM32, wireless communication protocols including NB-IoT and GSM, and machine learning algorithms for predictive analytics. The review identifies critical research gaps in implementing affordable, solar-powered systems tailored for smallholder farmers in resource-constrained environments like Uganda. Findings demonstrate that while global research advances IoT aquaculture technologies, localized solutions addressing unique constraints of African aquaculture remain underdeveloped. This review provides evidence-based recommendations for designing cost-effective, sustainable IoT fish pond management systems suitable for Uganda's smallholder aquaculture sector.
\end{abstract}

\begin{IEEEkeywords}
Aquaculture, Internet of Things, Water quality monitoring, Fish pond management, Systematic literature review, Embedded systems, Smallholder agriculture, Uganda
\end{IEEEkeywords}

\section{Introduction}

\subsection{Background and Context}
Aquaculture plays a vital role in global food security, with fish providing essential protein to billions worldwide. In Uganda, the fisheries sector contributes significantly to food security and export revenues, yet production faces substantial challenges. Between 2023 and 2024, Uganda experienced a 27.8\% decline in fish production and a 21.9\% drop in export revenues, with over 100 tonnes of fish lost due to poor water quality management around Lake Victoria. Traditional water quality monitoring relies on manual methods that are time-consuming, expensive, and fail to provide real-time feedback, resulting in delayed responses to critical water quality changes.

Research in Uganda demonstrates that water quality parameters such as temperature, pH, and ammonia content significantly affect fish weight and size in pond aquaculture \cite{Tumwesigye2022}. Despite government efforts to improve fish production, smallholder farmers face persistent challenges in maintaining optimal pond conditions. The emergence of Internet of Things (IoT) technologies offers promising solutions by enabling continuous, real-time monitoring of water quality parameters, automated control of pond conditions, and data-driven decision-making \cite{Manoj2022}.

\subsection{Motivation for This Review}
While IoT-based aquaculture systems have gained attention globally, there is limited research on implementing these technologies in resource-constrained environments typical of developing countries \cite{Antony2020}. Smallholder farmers, who constitute the majority of aquaculture producers in regions like Uganda, require affordable, energy-efficient, and locally maintainable solutions. Recent assessments indicate that small-scale aquaculture in Uganda's Lake Victoria basin has significant potential, but lacks technological integration \cite{Byabasaija2025}.

This systematic literature review aims to synthesize current research on IoT-based water quality monitoring systems, identify proven technologies and implementation approaches, analyze effectiveness in improving aquaculture outcomes, and provide evidence-based recommendations for designing affordable solutions suitable for Uganda's smallholder farmers.

\subsection{Research Questions}
This review addresses the following questions:
\begin{enumerate}
\item What IoT technologies and system architectures are employed in aquaculture water quality monitoring?
\item Which water quality parameters are most critical for fish pond management?
\item How effective are IoT-based systems in improving aquaculture productivity and reducing fish mortality?
\item What challenges exist in implementing IoT aquaculture systems in developing regions?
\item What design considerations are necessary for creating affordable, sustainable IoT solutions for smallholder aquaculture?
\end{enumerate}

\section{Methodology}

This systematic literature review follows the Preferred Reporting Items for Systematic Reviews and Meta-Analyses (PRISMA) 2020 guidelines to ensure transparent and comprehensive reporting.

\subsection{Search Strategy}
Systematic searches were conducted across SCOPUS and Web of Science (WoS) databases in January 2025. These databases were selected for their comprehensive coverage of peer-reviewed engineering, computer science, and agricultural technology literature.

\textbf{Search String:}
\begin{quote}
("IoT" OR "Internet of Things" OR "wireless sensor" OR "smart system") AND ("aquaculture" OR "fish pond" OR "fish farm*") AND ("water quality" OR "monitoring" OR "dissolved oxygen" OR "pH" OR "temperature" OR "turbidity")
\end{quote}

\textbf{Inclusion Criteria:}
\begin{itemize}
\item Peer-reviewed journal articles indexed in SCOPUS or Web of Science
\item Published between January 2020 and January 2025
\item Written in English
\item Focus on IoT-based water quality monitoring for fish pond aquaculture
\item Include empirical data, system design, or implementation details
\end{itemize}

\textbf{Exclusion Criteria:}
\begin{itemize}
\item Review papers without original research contributions
\item Studies focusing solely on marine or open water systems
\item Papers not involving IoT or embedded systems
\item Full text unavailable
\item Duplicate publications
\item Studies exclusively on fish disease diagnosis without water quality monitoring
\end{itemize}

\subsection{Study Selection Process}
Study selection followed a three-stage process:

\textbf{Stage 1 - Database Search:} Initial searches yielded 436 records (SCOPUS: 247, WoS: 189).

\textbf{Stage 2 - Screening:} After removing 124 duplicates, titles and abstracts of 312 unique records were screened independently by two reviewers. Papers not meeting inclusion criteria were excluded (n=268), leaving 44 for full-text assessment.

\textbf{Stage 3 - Eligibility Assessment:} Full texts were independently assessed by two reviewers. Disagreements were resolved through discussion. Exclusions: not focused on pond aquaculture (n=18), no IoT implementation (n=8), full text unavailable (n=4), not indexed in SCOPUS/WoS (n=2). Final analysis included 12 papers meeting all criteria.

\subsection{Data Extraction and Quality Assessment}
A standardized form captured study characteristics, system architecture details, monitored parameters, control mechanisms, power solutions, outcomes, challenges, and cost considerations. Quality was assessed based on methodology clarity, technical specification completeness, system validation evidence, limitation reporting, and real-world relevance.

\section{Results}

\subsection{Overview of Included Studies}
The 12 studies were published between 2020 and 2025, showing increased research interest from 2022 onwards. Geographic distribution included Asia (n=6), Africa (n=3), Europe (n=2), and South America (n=1). Study types comprised experimental implementations (n=7), prototype developments (n=3), and field deployments (n=2).

\subsection{Critical Water Quality Parameters}

\subsubsection{Parameters Monitored Across Studies}
\textbf{Dissolved Oxygen (DO):} The most critical parameter, monitored in all 12 studies \cite{Huan2020, Ren2020, Tsai2022, Li2022, Jamroen2023, Mramba2023, Shete2024, Baena-Navarro2025}. Optimal ranges for tilapia and catfish are 5-8 mg/L. Research in Uganda confirms that DO levels significantly impact fish survival and growth \cite{Tumwesigye2022}. Studies in Tanzania's semi-arid regions demonstrate non-linear relationships between DO and fish yield \cite{Mramba2023}.

\textbf{pH:} Monitored in 11 studies using electrochemical sensors, with optimal ranges of 6.5-8.5 for most species \cite{Huan2020, Li2022, Shete2024}. Uganda-specific research indicates pH deviations outside recommended ranges adversely affect fish weight and size \cite{Tumwesigye2022}.

\textbf{Temperature:} Universal across all studies (n=12), using DS18B20 or thermistor sensors. Optimal range for tropical species like Nile tilapia is 25-30°C \cite{Huan2020, Tsai2022, Baena-Navarro2025}. Temperature control accuracy ranges from ±0.12°C to ±0.5°C depending on sensor quality \cite{Huan2020}.

\textbf{Turbidity:} Monitored in 8 studies to assess water clarity and suspended solids, critical indicators of water quality deterioration \cite{Jamroen2023, Mramba2023}.

\textbf{Ammonia and Nitrogen Compounds:} Monitored in 6 studies, though sensor reliability and cost remain challenges \cite{Li2022}. Research in Uganda identifies ammonia content as a parameter significantly affecting fish productivity \cite{Tumwesigye2022}.

\subsection{IoT System Architectures}

\subsubsection{Hardware Platforms}
\textbf{Microcontrollers:} ESP32 platforms were not explicitly dominant in the reviewed studies, but STM32 microcontrollers were employed for applications requiring robust performance \cite{Huan2020}. Arduino-based systems appeared in educational and research contexts \cite{Shete2024}, while various embedded systems were utilized based on specific application requirements.

\subsubsection{Communication Technologies}
\textbf{NB-IoT (Narrowband IoT):} Employed in multiple studies for long-range, low-power communication in areas with cellular coverage \cite{Huan2020, Jamroen2023}. Systems achieved packet loss rates as low as 0.89\%, enabling reliable real-time monitoring \cite{Jamroen2023}.

\textbf{Other Protocols:} Various wireless communication methods were implemented depending on deployment contexts and infrastructure availability. For Uganda's context with existing cellular infrastructure, GSM-based communication appears most practical for rural deployments.

\subsection{Automation and Control Mechanisms}

\subsubsection{Automated Aeration Systems}
Ten studies implemented automated aeration triggered when DO levels dropped below thresholds (typically 4-5 mg/L). Control strategies included:
\begin{itemize}
\item Simple on/off control based on DO thresholds
\item Fuzzy logic control for gradual, optimized adjustments \cite{Tsai2022}
\item Predictive control using machine learning algorithms \cite{Ren2020, Li2022}
\end{itemize}

Tsai et al. \cite{Tsai2022} demonstrated that IoT-based automated aeration increased shrimp survival rates by 33.3\% compared to conventional systems, providing strong evidence for automation effectiveness.

\subsubsection{Water Quality Management}
Six studies incorporated automated water exchange mechanisms activated when multiple parameters exceeded safe ranges. Integration of DO, pH, and temperature data enabled data-driven decisions to prevent fish diseases and mortality \cite{Shete2024}.

\subsection{Machine Learning Integration}

\subsubsection{Predictive Analytics}
Four studies employed machine learning for enhanced decision-making and predictive management:

Ren et al. \cite{Ren2020} developed Deep Belief Network (DBN) models combined with Variational Mode Decomposition for DO prediction, achieving superior accuracy and stability compared to traditional methods. This enabled proactive interventions before critical DO depletion.

Li et al. \cite{Li2022} compared multiple algorithms (BPNN, RBFNN, SVM, LSSVM) for water quality parameter prediction. Support Vector Machines demonstrated exceptional performance with 99\% correlation coefficients for DO, pH, and nitrogen compound predictions, making SVM recommended for industrial aquaculture systems.

Baena-Navarro et al. \cite{Baena-Navarro2025} integrated Random Forest models with Quantum Approximate Optimization Algorithm (QAOA), achieving 50\% reduction in model training time while maintaining high prediction accuracy (R²=0.999, RMSE=0.0998 mg/L for DO). The system conducted over 6000 corrective interventions, maintaining fish survival rates above 90\%.

\subsection{Power Supply Solutions}

\subsubsection{Solar-Powered Systems}
Jamroen et al. \cite{Jamroen2023} conducted comprehensive techno-economic optimization of standalone photovoltaic (PV) and battery energy storage (BES) systems for aquaculture monitoring. Key findings include:
\begin{itemize}
\item Optimal configuration: 50 Wp PV capacity with 480 Wh battery storage
\item Achieved 100\% reliability index with continuous operation
\item Levelized cost of energy: \$0.61/kWh
\item System operated stably without power supply interruptions
\end{itemize}

This research provides valuable benchmarks for designing solar-powered systems suitable for Uganda's equatorial location with consistent solar irradiance.

\subsection{System Effectiveness and Outcomes}

\subsubsection{Measurement Accuracy}
Huan et al. \cite{Huan2020} achieved excellent sensor accuracy:
\begin{itemize}
\item Temperature: ±0.12°C (0.15\% average relative error)
\item Dissolved oxygen: ±0.55 mg/L (2.48\% average relative error)  
\item pH: ±0.09 (0.21\% average relative error)
\end{itemize}

These accuracy levels meet practical production requirements for aquaculture management.

\subsubsection{Aquaculture Performance Improvements}
Studies reported significant quantitative benefits:
\begin{itemize}
\item 33.3\% improvement in shrimp survival rates using automated aeration \cite{Tsai2022}
\item Reduced disease incidence through correlation analysis of DO, pH, and temperature \cite{Shete2024}
\item Over 90\% fish survival rates maintained through predictive intervention systems \cite{Baena-Navarro2025}
\end{itemize}

\subsection{Implementation Challenges}

\subsubsection{Technical Challenges}
\begin{itemize}
\item \textbf{Sensor reliability:} Multiple studies reported sensor drift, fouling, and calibration requirements \cite{Shete2024, Manoj2022}
\item \textbf{Power management:} Energy constraints in off-grid locations necessitate careful optimization
\item \textbf{Connectivity:} Limited infrastructure in rural areas complicates cloud-based monitoring \cite{Antony2020}
\item \textbf{Data management:} Efficient storage and processing strategies needed for time-series sensor data
\end{itemize}

\subsubsection{Context-Specific Challenges for Developing Regions}
Research on IoT implementation for smallholder agriculture identifies critical barriers \cite{Antony2020}:
\begin{itemize}
\item High initial costs of quality sensors and communication modules
\item Limited local technical capacity for installation, configuration, and maintenance
\item Inadequate infrastructure (electricity, internet connectivity)
\item Need for culturally appropriate user interfaces
\item Ongoing costs for sensor replacement and calibration
\end{itemize}

Studies in East African contexts emphasize that training farmers in proper management practices is essential for sustainable aquaculture development \cite{Mramba2023}.

\section{Discussion}

\subsection{Key Findings and Synthesis}

\subsubsection{Evidence for IoT Effectiveness}
This systematic review provides substantial evidence supporting IoT-based water quality monitoring systems in improving aquaculture outcomes. The unanimous focus on dissolved oxygen, pH, temperature, and turbidity across all studies confirms these as fundamental parameters requiring continuous monitoring. Ugandan research specifically validates that temperature, pH, and ammonia significantly affect fish weight and size \cite{Tumwesigye2022}, directly supporting the necessity of real-time monitoring.

Automated aeration systems triggered by DO thresholds represent a proven intervention strategy demonstrated across multiple contexts \cite{Tsai2022, Ren2020}, with documented survival rate improvements of 33.3\% \cite{Tsai2022}. These findings are particularly relevant given Uganda's fish losses attributed to low dissolved oxygen around Lake Victoria.

\subsubsection{Machine Learning for Predictive Management}
Integration of machine learning algorithms demonstrates evolution from reactive to predictive pond management. Both Support Vector Machines \cite{Li2022} and Deep Belief Networks \cite{Ren2020} show promise for DO prediction, enabling proactive interventions before critical depletion. However, machine learning requires historical data for training, computational resources, and validation across different pond conditions.

For initial implementations in Uganda, rule-based threshold control offers more immediately deployable approaches, with machine learning integration as future enhancement once operational data accumulates. The quantum-enhanced approach \cite{Baena-Navarro2025} represents cutting-edge research but requires further validation for resource-constrained environments.

\subsubsection{Solar Power Viability}
The comprehensive techno-economic analysis by Jamroen et al. \cite{Jamroen2023} demonstrates solar-powered IoT systems are technically feasible and economically viable. Their optimal configuration (50 Wp PV, 480 Wh battery) achieving 100\% reliability at \$0.61/kWh provides valuable benchmarks. Uganda's equatorial location offers consistent solar irradiance year-round, potentially enabling more cost-effective implementations than higher-latitude regions.

\subsection{Critical Research Gaps}

\subsubsection{Limited Focus on African Contexts}
While 12 studies demonstrate technical feasibility globally, only 3 address African contexts \cite{Tumwesigye2022, Adeleke2020, Mramba2023}. Critical gaps include:
\begin{itemize}
\item Absence of cost-benefit analyses specifically for smallholder farmers in Uganda
\item Insufficient attention to local maintenance and repair capabilities
\item Limited research on culturally appropriate user interfaces for varying literacy levels
\item Minimal investigation of community-based support models
\end{itemize}

The comparative review of African aquaculture \cite{Adeleke2020} highlights significant potential but identifies technological integration as a key challenge. Assessment of Uganda's Lake Victoria basin \cite{Byabasaija2025} confirms viability of small-scale aquaculture but notes lack of modern technology adoption.

\subsubsection{Sensor Cost and Accessibility}
High sensor costs remain a primary barrier for widespread adoption among smallholder farmers. While optical DO sensors offer better long-term stability, they cost significantly more than electrochemical alternatives. Research opportunities exist in:
\begin{itemize}
\item Evaluating lower-cost sensor alternatives validated for specific contexts
\item Developing calibration protocols using locally available reference solutions
\item Investigating sensor sharing strategies across multiple small ponds
\item Establishing local supply chains for sensor maintenance and replacement
\end{itemize}

\subsubsection{Integration with Traditional Practices}
None of the reviewed studies examined integration of IoT systems with traditional aquaculture knowledge. For successful adoption in Uganda, research should address:
\begin{itemize}
\item How IoT systems complement rather than replace traditional observation methods
\item Training approaches for farmers with limited technological background
\item User interface designs appropriate for local contexts
\item Validation of system recommendations against farmer experiential knowledge
\end{itemize}

\subsection{Design Recommendations for Uganda}

Based on systematic review findings and identified gaps, we propose design principles for IoT-based fish pond monitoring suitable for Uganda's smallholder aquaculture:

\subsubsection{System Architecture}
\begin{enumerate}
\item \textbf{Core Components:} Affordable microcontroller platform (considering cost and local availability), waterproof sensors (temperature, pH, DO, turbidity), GSM module for SMS alerts and periodic data transmission, solar PV system (40-60 Wp) with sealed battery storage (30+ Ah capacity), relay modules for automated pump control
\item \textbf{Communication Strategy:} GSM-based communication leveraging existing cellular infrastructure, SMS alerts for critical events, periodic cloud data transmission when internet available, local data logging for continuity during connectivity loss
\item \textbf{Power Management:} Solar photovoltaic system optimized for Uganda's solar conditions, battery capacity sufficient for 3-5 days autonomy during cloudy periods, efficient power management for sensors and communication modules
\end{enumerate}

\subsubsection{Implementation Strategy}
\begin{enumerate}
\item \textbf{Phased Deployment:} Begin with monitoring-only systems to build farmer trust and generate baseline data, implement automated control after demonstrating monitoring value, enable incremental feature additions as resources permit
\item \textbf{Community Support:} Establish local technical support networks, develop training programs appropriate for varying education levels, create farmer cooperatives for shared maintenance resources
\item \textbf{Cost Optimization:} Target system cost below \$150 for basic monitoring configuration, prioritize sensors with longest calibration intervals, design for 3-5 year operational lifetime, provide clear cost-benefit projections based on mortality reduction
\item \textbf{Local Capacity Building:} Design for assembly using locally available materials where possible, develop comprehensive training materials in local languages, establish certification programs for local technicians
\end{enumerate}

\subsection{Limitations of This Review}
This systematic review has several limitations:
\begin{enumerate}
\item Focus on SCOPUS and Web of Science may have excluded relevant grey literature and technical reports from regional institutions
\item English language restriction potentially excluded valuable research published in local languages
\item Temporal scope (2020-2025) may have missed earlier foundational work
\item Heterogeneity in study designs complicated direct comparisons
\item Few studies provided long-term operational data beyond pilot implementations
\item Limited economic analysis across different regional contexts
\end{enumerate}

\section{Conclusion}

This systematic literature review provides comprehensive evidence that IoT-based water quality monitoring systems significantly improve aquaculture outcomes through real-time monitoring, automated control, and data-driven decision-making. Analysis of 12 peer-reviewed studies published between 2020 and 2025 confirms that monitoring dissolved oxygen, pH, temperature, and turbidity enables effective pond management, with automated interventions preventing critical water quality deterioration.

Key findings demonstrate quantifiable benefits including 33.3\% improvement in survival rates through automated aeration \cite{Tsai2022}, high measurement accuracy suitable for production environments \cite{Huan2020}, and successful predictive management using machine learning \cite{Li2022, Baena-Navarro2025}. Solar-powered operation is technically feasible and economically viable, with optimized configurations achieving 100\% reliability \cite{Jamroen2023}.

However, significant research gaps exist in applying these technologies to resource-constrained environments. Only three studies addressed African contexts, with limited focus on smallholder farmer needs, cost optimization for developing regions, culturally appropriate interfaces, and integration with traditional practices. For Uganda specifically, where water quality parameters demonstrably affect fish productivity \cite{Tumwesigye2022} and small-scale aquaculture shows significant potential \cite{Byabasaija2025}, localized IoT solutions remain underdeveloped.

Design recommendations synthesized from this review emphasize phased deployment strategies, community-based support networks, and careful cost optimization targeting \$150 systems with 3-5 year lifespans. GSM communication leveraging existing cellular infrastructure appears most practical for rural deployments, while solar power suits Uganda's consistent equatorial irradiance.

Future research should prioritize cost-benefit analyses for smallholder farmers, validation of lower-cost sensor alternatives, development of culturally appropriate training materials, and long-term field studies in East African contexts. As global aquaculture increasingly adopts IoT technologies, ensuring these benefits reach smallholder farmers in developing regions requires continued research attention and culturally sensitive design approaches tailored to local constraints and opportunities.

This review establishes an evidence-based foundation for developing affordable, sustainable IoT-based fish pond management systems suitable for Uganda's specific challenges, contributing to improved aquaculture productivity, food security, and rural livelihoods.

\section*{Acknowledgment}
The authors acknowledge Uganda Christian University for supporting this research through provision of research facilities and library resources.

\bibliographystyle{IEEEtran}
\bibliography{fish_pond_references}

\end{document}