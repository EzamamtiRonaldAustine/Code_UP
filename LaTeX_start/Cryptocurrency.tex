\documentclass[11pt]{article}
\usepackage{graphicx}
\usepackage{amsmath}
\usepackage{hyperref}
\hypersetup{
  colorlinks=true,
  linkcolor=black,
  citecolor=black,
  filecolor=black,
  urlcolor=black
}
\usepackage{indentfirst}
\usepackage{geometry}
\usepackage{tcolorbox}
\usepackage{longtable}
\usepackage{fancyhdr}
\usepackage{xcolor}
\usepackage{tikz}  
\usepackage{setspace}
\geometry{a4paper, margin=1in,top=1in}
% Set line spacing to 1.5
\onehalfspacing
% Set up the header and footer
\pagestyle{fancy}
\renewcommand{\headrulewidth}{2pt}
\renewcommand{\footrulewidth}{2pt}

% color
\definecolor{darkbrown}{rgb}{0.5, 0.18, 0.10}
\fancyhf{} % Clear default header and footer content

% Define header contents
\fancyhead[L]{Volume: 6 No.3} % Left side
\fancyhead[C]{October 2021} % Center
\fancyhead[R]{ISSN: 2455-5398} % Right side

% Set the footer
% Define footer contents (text remains black)
\fancyfoot[L]{\textbf{\textcolor{black}{National Journal of Research in Marketing, Finance \& HRM}}} % Black text
\fancyfoot[R]{\textcolor{black}{\thepage}} % Black page number

% Change only the footrule color to brown
\makeatletter
\renewcommand{\headrule}{\color{darkbrown}\hrule height \footrulewidth \@width \headwidth \vskip 2pt}
\makeatother

\makeatletter
\renewcommand{\footrule}{\color{darkbrown}\hrule height \footrulewidth \@width \headwidth \vskip 2pt}
\makeatother


\begin{document}

% Title Page
\begin{titlepage}
    \centering
    \vspace*{1cm}
    % Logo Placeholder
    \includegraphics[width=15cm]{img/UCU.png} \\[1.5cm] % Adjust width as needed
    \Huge
    \textbf{Pros and Cons of Cryptocurrency: A Brief Overview Redesigned in Latex for Learning Purposes}
    
    \vspace{0.5cm}
    \LARGE
    \textbf{Dr. Kishor P. Bholane} \\
    Head, Department of Commerce \\
    Vinayakrao Patil Mahavidyalaya, Vaijapur \\
    Email ID: kishor\_bholane@rediffmail.com
    
    \vspace{1cm}

    \textbf{By}
    
    \vspace{0.5cm}
    
    \begin{tabular}{|c|c|c|}
        \hline
        \textbf{Name} & \textbf{Reg. No} & \textbf{Access No} \\ \hline
        Anna Akumu Emokol & S23B23/094 & B24782 \\ \hline
        Ezamanti Ronald Austine & S23B23/018  & B24252 \\ \hline
        Nankya Zaharah & S23B23/082 & B24771 \\ \hline
    \end{tabular}
    
    \vfill
    
\end{titlepage}
\title{\textbf{Pros and Cons of Cryptocurrency: A Brief Overview}}
\author{\textbf{Dr. Kishor P. Bholane} \\ Head, Department of Commerce \\ Vinayakrao Patil Mahavidyalaya, Vaijapur \\ Email ID: kishor\_bholane@rediffmail.com}
\date{}
\maketitle
\thispagestyle{fancy}

\setcounter{page}{71}


\begin{tcolorbox}[colback=light gray, colframe=black, title=Abstract, sharp corners=south]
The business world is seeing towards the cryptocurrency as a future currency. A very less literature is available on cryptocurrencies. This research paper focused on the concept, features, history and the mechanism of cryptocurrency. It also discussed the current status of cryptocurrencies in India and some leading cryptocurrencies with their market cap. While considering cryptocurrency as a digital investment, its pros and cons are to be kept in mind, which are also included in this research paper.  
\\
\\
\textbf{Keywords:} Cryptocurrency, Block Chain Technology, Bitcoin, Cryptograph etc.
\end{tcolorbox}

\section*{Introduction:}
Cryptocurrencies fall under the banner of digital currencies, alternative currencies and virtual currencies. They were initially designed to provide an alternative payment method for online transactions. However, cryptocurrencies have not yet been widely accepted by businesses and consumers and they are currently too volatile to be suitable as methods of payment. Cryptocurrencies differ significantly from traditional fiat currencies. Nonetheless, one can still buy and sell them like any other asset. One can now also trade on the price movements of various cryptocurrencies via CFDs and spread betting. Cryptocurrencies use cryptography to secure transactions and regulate the creation of additional units. Bitcoin, the most well-known cryptocurrency, was launched in January 2009. Today there are near about 6000 cryptocurrencies available online.

\section*{Objectives of the Study:}
The objectives of this research paper are as follows:
\begin{enumerate}
    \item To understand the concept, nature and history of cryptocurrency.
    \item To make comparative analysis of fiat currency, money, assets and cryptocurrency.
    \item To overview some leading cryptocurrencies with their market cap.
    \item To understand the Mechanism of cryptocurrency.
    \item To study the legality and its current status in India.
    \item To study the pros and cons of cryptocurrencies as a digital investment.
\end{enumerate}


\section*{Research Methodology of the Study:}
The concept of cryptocurrency is in its evolutionary phase, which needs further developments. This research paper is based on secondary data, which is obtained from various journals and websites trading in cryptocurrencies. This research paper is intended to add literature on the concept, features, its working, pros and cons of cryptocurrencies.

\section*{What is Cryptocurrency?}
\indent Cryptocurrency refers to the technology that acts as a medium for facilitating the conduct of the different financial transactions which are safe and secure and it is one of the tradable digital forms of the money allowing the person to send or receive the money from the other party without any help of the third party service.

\section*{Features of Cryptocurrency:}
\begin{enumerate}
    \item Cryptocurrencies work through decentralized platforms and are based on blockchain technology.
    \item Cryptocurrencies are very secure. They are secured by the cryptography codes.
    \item Transactions in cryptocurrencies are irreversible.
    \item Another great feature of it is that they are super-fast. After initiating a transaction, it is immediately caught up in the network and it is confirmed just within two minutes of time.
    \item Cryptocurrencies don’t care about the owner’s physical location.
    \item As assets, cryptocurrencies are generally stored in digital wallets, which allow users to manage and trade their coins.
    \item They are created using a distributed ledger and peer-to-peer review.
\end{enumerate}

\section*{History of Cryptocurrency:}
The story of virtual coins begins with the cryptographer David Chaum. In 1983, he developed a cryptographic system called eCash. Twelve years later, he developed another system, DigiCash, which used cryptography to make economic transactions confidential. However, the first time the idea or term "cryptocurrency" was coined in 1998. That year, Wei Dai began to think about developing a new payment method that used a cryptographic system and whose main characteristic was decentralization. In 2009, Satoshi Nakamoto, a person whose identity is still secret, created the first cryptocurrency, Bitcoin. His intention to create a new way of payment was that could be used internationally, decentralized and without having any financial institution behind it. In October 2011, Litecoin was released. In late 2012, WordPress became the first major merchant to accept payment in Bitcoin. Many companies are now allowing the payment of their products and services with these virtual currencies and they even created their own. In June 2021, El Salvador became the first country to accept Bitcoin as legal tender. In August 2021, Cuba followed with Resolution 215 to accept Bitcoin as legal tender. There is a sharp rise in the number of cryptocurrencies from 66 in 2013 to 5840 in August 2021.


\begin{figure}[h]
    \centering
    \caption{Number of cryptocurrencies worldwide from 2013 to August 2021}
    \includegraphics[width=15cm]{img/bar.png} \\[0.5cm] % Adjust width as needed
    % \label{fig:cryptocurrencies}
    \textbf{Source:} \url{ https://www.statista.com/statistics/863917/number-crypto-coins-tokens/}
\end{figure}


\section*{Cryptocurrencies, Money and Assets:}
Cryptocurrencies have some of the characteristics of financial assets and fiat money, but today they cannot be equated with them. Many countries are actively promoting the development of payments using cryptocurrencies, so digital currencies are already partially a means of payment. 
Comparative characteristics of digital currencies as money and financial assets are given below: 
\begin{longtable}[c]{|l|l|l|l|}
    \caption{\textbf{Comparative Analysis of Cryptocurrencies, Money and Assets}} \\
    \hline
    \textbf{Characteristic} & \textbf{Fiat Money} & \textbf{Assets} & \textbf{Cryptocurrency} \\ \hline
    \endfirsthead

    \hline
    
    \endhead

    \hline
    \endfoot

    Medium of Exchange & Yes & No & Partially \\ \hline
    Unit of Account & Yes & No & No \\ \hline
    Property Right & No & Yes & Possible \\ \hline
    Economic Benefits from Ownership & Possible & Yes & Possible \\ \hline
    Is Liability from a Third Party & Yes & No & No \\ \hline
    Information Transfer and Storage Function & No & No & Yes \\ \hline
\end{longtable}
\noindent Perspective. Journal of Open Innovation: Technology, Market and Complexity. \\
\indent Cryptocurrencies have one important feature that distinguishes them from fiat money and 
financial assets: they have a unique function for processing massive amounts of information. 


\section*{Most Common Types of Cryptocurrency:}
When Bitcoin launched in 2009, it didn’t have much or any competition. By 2011, though, new types of cryptocurrency began to emerge as competitors. Today there are thousands of different types of cryptocurrency. Here is a list of the 10 biggest cryptocurrencies by market capitalization (as on 14 September 2021).Because there are so many virtual currencies, market cap helps to identify those with the highest valuation. 

\begin{enumerate}
    \item \textbf{Bitcoin}: The first cryptocurrency created in 2009. It was designed to be independent 
    of any government or central bank and relies on blockchain technology. There are more than 
    18.8 million Bitcoin tokens in circulation as of September 2021, against a capped limit of 21 
    million.
    \item \textbf{Ethereum}: Like Bitcoin, Ethereum is a blockchain network. ETH is also generated using a 
    proof-of-work system. But unlike Bitcoin, there is no limit to the number of ETHs that can be 
    created.
    \item \textbf{Cardano (ADA)}: Cardano is a third-generation blockchain platform. Cardano relies on proof-of
    stake (PoS), meaning that the complicated PoW calculations and high electricity usage required 
    for mining coins like Bitcoin aren’t necessary, potentially making its network more efficient and 
    sustainable. Its cryptocurrency is called ADA.  
    \item \textbf{Binance Coin (BNB)}:  BNB was created as a utility token in 2017 that allowed traders to get 
    discounts on trading fees on Binance, but now it can also be used for payments. Every quarter, 
    Binance buys back and then “burns” or permanently destroys some of the coins it holds to drive 
    demand.
    \item \textbf{Tether}: Tether was the first cryptocurrency marketed as a stablecoin. Like other stablecoins, the 
    tether is designed to offer stability, transparency, and lower transaction charges to users. Tether 
    is pegged to the U.S. dollar. 
    \item \textbf{Solana}: Solana is a blockchain platform that generates the cryptocurrency known as Sol. One of 
    the more volatile currencies, the Sol was trading at about \$191.00 on September 10, 2021 and 
    one year ago it was worth \$3.42. 
    \item \textbf{XRP}: XRP was developed by Ripple Labs. Unlike Bitcoin and many other cryptocurrencies, 
    XRP can’t be mined; instead there are a limited number of coins. The Ripple network employs a 
    unique system for validating transactions which, makes XRP transactions faster and cheaper than 
    Bitcoin.
    \item \textbf{Dogecoin}:  Dogecoin is similar to Bitcoin and Ethereum in that it’s run on a blockchain network 
    using a PoW system. But the number of coins that can be mined are unlimited (versus the 21 
    million-coin cap on Bitcoin). It was launched in 2013. 
    \item \textbf{Polkadot (DOT)}:Polkadot was co-founded by Gavin Wood, also a co-founder of Ethereum, to 
    take the capabilities of a blockchain network to another level. What differentiates Polkadot from 
    other blockchains is its core mission to solve the problem of interoperability by building so- 
    called bridges between blockchains. 
    \item \textbf{USD (USDC)}: USD Coin (USDC) is a stablecoin that runs on the Ethereum blockchain and 
    several others. A USDC is worth one U.S. dollar. The guaranteed 1:1 ratio, making it a stable 
    form of exchange. The goal of having a stablecoin like USDC is to make transactions faster and 
    cheaper.
\end{enumerate}

% \section*{Market Cap of Some Leading Cryptocurrencies (As on 19 September 2021)}
\begin{table}[h]
    \centering
    \caption{\textbf{Market Cap of Some Leading Cryptocurrencies }(As on 19 September 2021)}
    \renewcommand{\arraystretch}{1.8} % Increases row height for better spacing
    \setlength{\tabcolsep}{11pt} % Increases column separation for better readability
    \begin{tabular}{|c|l|l|l|l|}
        \hline
        \textbf{Rank} & \textbf{Name} & \textbf{Symbol} & \textbf{Market Cap} & \textbf{Price} \\ \hline
        1 & Bitcoin & BTC & \$889,472,404,519.70 & \$47,260.22 \\ \hline
        2 & Ethereum & ETH & \$391,520,172,702.19 & \$3,329.45 \\ \hline
        3 & Cardano & ADA & \$73,092,414,722.89 & \$2.28 \\ \hline
        4 & Binance Coin & BNB & \$68,679,211,877.59 & \$408.47 \\ \hline
        5 & Tether & USDT & \$68,322,670,124.26 & \$1.00 \\ \hline
        6 & XRP & XRP & \$48,880,060,223.91 & \$1.05 \\ \hline
        7 & Solana & SOL & \$45,292,491,129.39 & \$152.52 \\ \hline
        8 & Polkadot & DOT & \$33,381,499,814.59 & \$33.80 \\ \hline
        9 & Dogecoin & DOGE & \$30,623,676,761.57 & \$0.2331 \\ \hline
        10 & USD Coin & USDC & \$29,443,458,704.74 & \$1.00 \\ \hline
    \end{tabular}
    
    \textbf{Source:} \url{https://www.investing.com/crypto/currencies}
\end{table}


\section*{Mechanism of Cryptocurrency:}
Cryptocurrency runs on blockchain technology. A blockchain is simply a digital ledger of transactions. This ledger is distributed across a network of computer systems. No single system controls the ledger. Instead, a decentralized network of computers keeps a blockchain running and authenticates its transactions. Cryptocurrency transactions are recorded in perpetuity on the underlying blockchain. Groups of transactions are added to the ‘chain’ in the form of ‘blocks,’ which validate the authenticity of the transactions and keep the network up and running. All batches of transactions are recorded on the shared ledger, which is public. Anyone can go and look at the transactions being made on the major blockchains, such as Bitcoin (BTC) and Ethereum (ETH). \\
\indent The computers ‘working’ to ‘prove’ the authenticity of blockchain transactions are known as miners. In return for their energy, miners receive freshly minted crypto assets. Investors in cryptocurrencies don’t hold their assets in traditional bank accounts. Instead, they have digital addresses. These addresses come with private and public keys - long strings of numbers and letters - that enable cryptocurrency users to send and receive funds. Private keys allow cryptocurrency to be unlocked and sent. Public keys are publicly available and enable the holder to receive cryptocurrency from any sender.

\section*{Current Status of Cryptocurrencies in India:}
Cryptocurrencies are growing popular among Indian investors also. These currencies basically disrupt the central bank model of transaction and trading. Indians have invested over \$6 million in cryptocurrencies. Almost 1.5 crore Indians are estimated to have made these investments. Several hundred start-ups have also started operating in the blockchain and cryptocurrency space. Cryptocurrencies are not illegal in India. So if you want to buy, let's say Bitcoins, you can do so and start trading in it. However, India does not have a regulatory framework to govern cryptocurrencies as of now. \\
\indent The government had constituted an Inter-Ministerial Committee (IMC) on November 2, 2017, to study virtual currencies. The Group's report, along with a Draft Bill, flagged the positive aspect of distributed-ledger technology and suggested various applications, especially in financial services, for its use in India, including banks and other financial firms. In April 2018, the Reserve Bank of India (RBI) brought out an advisory notice to "all the entities regulated by it not to deal in" cryptocurrency. However, in 2020, the Supreme Court lifted the central bank's ban and allowed banks and financial institutions to provide services even if they were related to digital currencies. In May 2021, the Deputy Governor of RBI said that they were considering ways to bring out a Central Bank Digital Currency (CBDC). The cryptocurrency bill could facilitate its coming. A much-awaited cryptocurrency bill in this regard has been tabled before the Union Cabinet and awaits its approval.

\section*{The Pros and Cons of Cryptocurrency as a Digital Investment:}
Cryptocurrency, both conceptually and in practice, is very complicated. Most people who own cryptocurrency don’t really understand how the underlying principles of cryptography work. Before investing in cryptocurrencies it is necessary to keep in mind the pros and cons of cryptocurrencies.

\subsection*{Pros of Cryptocurrency:}
\begin{enumerate}
    \item Cryptocurrency networks are inherently secure.
    \item Mining is accessible to anyone.
    \item Price fluctuations can create huge profits.
    \item Cryptocurrency behaves like real currency.
    \item Cryptocurrency is a potential hedge against inflation.
    \item Crypto transactions are generally cheaper than traditional electronic financial transactions.
    \item Cryptocurrency can reduce the cost of international transactions.
\end{enumerate}

\subsection*{Cons of Cryptocurrency:}
\begin{enumerate}
    \item Cryptocurrencies are extremely volatile.
    \item Cryptocurrency is less liquid than fiat currency or the stock market.
    \item In previous years, cryptocurrencies have been taxed at a relatively generous rate and governments have been largely hands-off. The laws around cryptocurrencies are continuously changing.
    \item There is no recourse for digital asset recovery.
    \item Making serious money from mining requires a commitment of time and money.
    \item Cryptocurrency has high potential for tax evasion in some jurisdictions.
    \item Lack of regulation facilitates black market activity.
\end{enumerate}

\section*{Concluding Remark:}
Cryptocurrencies are an outstanding technological innovation of the decade. The blockchain technology makes decentralized operations secured. There is a need for the government and regulating authorities to study and understand the working of a cryptocurrency. Cryptocurrency organizations along with the policy makers can create a significant and secure currency exchange. The legitimate use of Bitcoin or other cryptocurrencies in near future is doubtful, but the use of blockchain technology certainly has a long way to go. While considering cryptocurrency as a digital investment avenue, its pros and cons of cryptocurrencies must be kept in mind. Before investing in any cryptocurrency, the stand of Indian government regarding the legality and its regulation should be taken into account.

\section*{References}
\begin{enumerate}
    \item Dostov, Victor and Pavel Shust (2014). Cryptocurrencies: an unconventional challenge to the AML/CFT regulators? Journal of Financial Crime, Vol. 21 (3), pp. 249-263.
    \item Extance Andy (2015). The future of cryptocurrencies: Bitcoin and beyond. DOI: 10.1038/526021.
    \item Halaburda Hanna (2016). Digital Currencies: Beyond Bitcoin. Communications \& Strategies. Vol. (103), pp. 77-92,213.
    \item Harwick Cameron (2016). Cryptocurrency and the Problem of Intermediation. The Independent Review, Vol. 20 (4), pp. 569-588.
    \item Rose Chris (2015). The Evolution of Digital Currencies: Bitcoin, A Cryptocurrency Causing A Monetary Revolution. The International Business \& Economics Research Journal (Online), Vol. 14 (4), p. 617.
    \item \url{https://www.investopedia.com/tech/most-important-cryptocurrencies-other-than-bitcoin/}
    \item \url{https://www.statista.com/statistics/863917/number-crypto-coins-tokens/}
    \item \url{https://www.investing.com/crypto/currencies}
    \item \url{https://www.moneycrashers.com/cryptocurrency-pros-cons-volatility-technology/}
    \item \url{https://www.sofi.com/learn/content/understanding-the-different-types-of-cryptocurrency/}
    \item \url{https://www.moneycrashers.com/best-bitcoin-crypto-wallets/}
    \item \url{https://www.nasdaq.com/articles/what-is-cryptocurrency-and-how-it-works}
    \item \url{https://economictimes.indiatimes.com/markets/cryptocurrency/class/articleshow/}
    \item \url{https://www.businesstoday.in/cryptocurrency-in-india-what-the-govt-stand-legal-status}
\end{enumerate}

\end{document}