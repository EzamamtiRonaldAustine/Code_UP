\documentclass[12pt,a4paper]{report}
\usepackage[utf8]{inputenc}
\usepackage[margin=1in]{geometry}
\usepackage{setspace}
\usepackage{etoolbox}
\usepackage{titlesec}
\usepackage{fancyhdr}
\usepackage{graphicx}
\usepackage{amsmath}
\usepackage{amsfonts}
\usepackage{amssymb}
\usepackage{hyperref}
\hypersetup{
  colorlinks=true,
  linkcolor=black,        % Keep internal links black
  citecolor=black,        % Keep citation links black  
  filecolor=black,        % Keep file links black
  urlcolor=blue          % Change URL links to blue
}
\usepackage{multicol}
\usepackage{array}
\usepackage{longtable}
\usepackage{multirow}
\usepackage{booktabs}
\usepackage{chngcntr}
\usepackage{float}
\usepackage{grffile}
\usepackage{tocloft}
\usepackage{ragged2e}
\usepackage{titletoc}  % Added for better TOC control

% Set Times New Roman as the main font using more robust method
\usepackage{newtxtext,newtxmath} % Better Times New Roman with proper bold
\usepackage[T1]{fontenc} % Better font encoding for proper bold rendering

% Page setup
\setstretch{1.5}
\pagestyle{fancy}
\fancyhf{}
\renewcommand{\headrulewidth}{0pt}
\fancyfoot[C]{\thepage}

\geometry{margin=1in}
\justifying

% Remove chapter number from figure numbering and use whole numbers
\counterwithout{figure}{chapter}
\renewcommand{\thefigure}{\arabic{figure}}

% Chapter title formatting - 16pt bold for numbered chapters
\titleformat{\chapter}[display]
  {\normalfont\fontsize{16}{19.2}\bfseries\centering}{}{0pt}{\centering}
\titlespacing{\chapter}{0pt}{-40pt}{20pt}

% Custom command for preliminary pages with 14pt font
\newcommand{\prelimchapter}[1]{%
  \chapter*{#1}%
  \addcontentsline{toc}{chapter}{#1}%
  \markboth{#1}{#1}%
}

% Section title formatting - 14pt bold
\titleformat{\section}
  {\normalfont\fontsize{14}{16.8}\bfseries}{\thesection}{0.5em}{}
\titlespacing{\section}{0pt}{15pt}{10pt}

% Subsection title formatting - 12pt bold
\titleformat{\subsection}
  {\normalfont\fontsize{12}{14.4}\bfseries}{\thesubsection}{0.5em}{}
\titlespacing{\subsection}{0pt}{10pt}{5pt}

% Subsubsection title formatting - 12pt bold
\titleformat{\subsubsection}
  {\normalfont\fontsize{12}{14.4}\bfseries}{\thesubsubsection}{0.5em}{}
\titlespacing{\subsubsection}{0pt}{8pt}{3pt}

% Table of Contents formatting
\renewcommand{\contentsname}{Table of Contents}
\setlength{\cftbeforechapskip}{5pt}
\setlength{\cftbeforesecskip}{2.5pt}
\setlength{\cftbeforesubsecskip}{1.5pt}

% Reduce spacing before and after TOC title
\setlength{\cftbeforetoctitleskip}{-20pt}
\setlength{\cftaftertoctitleskip}{10pt}

% Add leader dots for all TOC entries
\renewcommand{\cftchapleader}{\cftdotfill{\cftdotsep}}
\renewcommand{\cftsecleader}{\cftdotfill{\cftdotsep}}
\renewcommand{\cftsubsecleader}{\cftdotfill{\cftdotsep}}
\renewcommand{\cftsubsubsecleader}{\cftdotfill{\cftdotsep}}

% Ensure chapter entries have dots too
\renewcommand{\cftchapaftersnum}{}

% Format TOC title to 14pt
\renewcommand{\cfttoctitlefont}{\normalfont\fontsize{14}{16.8}\bfseries\centering}

% Indent sections and subsections in TOC
\setlength{\cftsecindent}{1.5em}
\setlength{\cftsubsecindent}{3.0em}
\setlength{\cftsubsubsecindent}{4.5em}

% List of Figures formatting
\renewcommand{\listfigurename}{List of Figures}
\setlength{\cftfigindent}{0pt}
\setlength{\cftfignumwidth}{3em}

% Reduce spacing before and after List of Figures title
\setlength{\cftbeforeloftitleskip}{-20pt}
\setlength{\cftafterloftitleskip}{10pt}

% Format List of Figures title to 14pt
\renewcommand{\cftloftitlefont}{\normalfont\fontsize{14}{16.8}\bfseries\centering}

% Remove chapter numbering patches
\makeatletter
\patchcmd{\@makechapterhead}{\vspace*{50\p@}}{\vspace*{0pt}}{}{}
\patchcmd{\@makeschapterhead}{\vspace*{50\p@}}{\vspace*{0pt}}{}{}
\makeatother

\begin{document}

\pagenumbering{roman}

% Title Page - Updated UIRI Version
\begin{titlepage}
    \centering
    \vspace*{-0.5cm}
    
    % UIRI Logo
    \makebox[\textwidth]{\includegraphics[width=18cm]{img/UIRI4.png}}\par
    \vspace{1.5cm}
    
    {\fontsize{18}{21.6}\selectfont\bfseries INTERNSHIP REPORT}\\[0.5cm]
    
    {\fontsize{14}{16.8}\selectfont\bfseries On}\\[0.4cm]
    
    {\fontsize{16}{19.2}\selectfont\bfseries ADVANCED HANDS-ON TRAINING IN EMBEDDED SYSTEMS AND INDUSTRIAL AUTOMATION}\\[1.2cm]
    
    {\fontsize{14}{16.8}\selectfont\bfseries By:}\\[0.4cm]
    {\fontsize{16}{19.2}\selectfont\bfseries EZAMAMTI RONALD AUSTINE}\\[0.3cm]
    {\fontsize{12}{16.8}\selectfont Reg. No: S23B23/018}\\[0.2cm]
    {\fontsize{12}{16.8}\selectfont Email: austineblackezamati@gmail.com}\\[1.2cm]
    
    {\fontsize{14}{16.8}\selectfont\bfseries A Report Submitted to the Training and Development Centre (TDC)}\\[0.3cm]
    {\fontsize{14}{16.8}\selectfont\bfseries Uganda Industrial Research Institute}\\[0.5cm]
    
    {\fontsize{12}{16.8}\selectfont\bfseries In Partial Fulfillment of the Requirements for the Award of the Degree of}\\[0.2cm]
    {\fontsize{12}{16.8}\selectfont\bfseries Bachelor of Science in Computer Science}\\[1.5cm]
    
    {\fontsize{14}{16.8}\selectfont\bfseries Period of Internship:}\\[0.3cm]
    {\fontsize{12}{16.8}\selectfont\textit{22\textsuperscript{nd} May 2025 - 18\textsuperscript{th} July 2025}}\\[1cm]

\end{titlepage}

% Declaration
{\fontsize{14}{16.8}\selectfont\bfseries\centering DECLARATION\par}
\vspace{10pt}
\addcontentsline{toc}{chapter}{Declaration}
\vspace{5pt}
\noindent I, \textbf{Ezamamti Ronald Austine}, certify that this internship report is the result of my original work and effort carried out at the \textbf{Uganda Industrial Research Institute (UIRI)}. In the preparation of this report, I have observed the provisions of the Uganda Christian University Code of Ethics.

\noindent I further affirm that the work is free from plagiarism, and all referenced materials have been duly acknowledged and attributed. I agree to indemnify and hold harmless Uganda Christian University from any claims that may arise due to copyright violations related to this report.

\vspace{2cm}
\noindent \textbf{Student's Signature:} \rule{6cm}{0.2pt} \hfill \textbf{Date:} \rule{3cm}{0.2pt}

\newpage
% Approval Page for UIRI
{\fontsize{14}{16.8}\selectfont\bfseries\centering APPROVAL\par}
\vspace{10pt}
\addcontentsline{toc}{chapter}{Approval}

\noindent This internship report has been reviewed and approved by the industrial supervisor named \\below.\\[1.2cm]

\noindent \textbf{Industrial Supervisor:} \\
Engineer. Musana Sabiiti John \\
Training and Development Centre (TDC) \\Uganda Industrial Research Institute (UIRI)\\[0.8cm]

\noindent Signature: \rule{5cm}{0.2pt} \hfill Date: \rule{3cm}{0.2pt}

\newpage
% Acknowledgement
{\fontsize{14}{16.8}\selectfont\bfseries\centering ACKNOWLEDGEMENT\par}
\vspace{10pt}
\addcontentsline{toc}{chapter}{Acknowledgement}
\noindent First and foremost, I thank the Almighty God for the wisdom, health, strength, and perseverance He granted me throughout this internship journey.

\noindent I extend my sincere gratitude to the entire management and staff of \textbf{Uganda Industrial Research Institute (UIRI)} for offering me the opportunity to acquire valuable advanced training. I am especially grateful to my industrial supervisor \textbf{Mr. Ngobi Elijah} for his unwavering guidance, mentorship, and commitment to ensuring meaningful learning experiences in embedded systems and professional development.

\noindent I also wish to express my appreciation to the following technical instructors whose expertise greatly enriched my technical growth:
\begin{itemize}
    \item \textbf{Mr. Atwine Philip}
    \item \textbf{Mr. Adaon Michel}
    \item \textbf{Mr. Mutesasira Bashir}
    \item \textbf{Mr. Ainebyona Savior}
    \item \textbf{Ms. Annah Faith}
\end{itemize}

\noindent Special thanks to \textbf{Dr. Innocent Ndibatya}, Head of the Department of Computing and Technology, \textbf{Mr. Isaac Ndawula}, my university supervisor, \textbf{Mr. Christopher Ssemambo}, and \textbf{Ms. Diana Nansubuga} at Uganda Christian University for ensuring this internship was possible and for providing academic support throughout the period.

\noindent I am also thankful to my fellow interns for the team spirit, knowledge sharing, and friendship that made this internship both productive and enjoyable.

\newpage
% Abstract
{\fontsize{14}{16.8}\selectfont\bfseries\centering ABSTRACT\par}
\vspace{10pt}
\addcontentsline{toc}{chapter}{Abstract}
\noindent This report documents a nine-week internship conducted at the Uganda Industrial Research Institute (UIRI) from May 22 to July 18, 2025, focused on practical skills development in embedded systems, electronics, and industrial automation. The internship offered structured exposure to modern engineering technologies and industrial practices under the supervision of experienced UIRI technical personnel.

\noindent The training followed a progressive curriculum starting with digital electronics foundations and analog circuit analysis, advancing through embedded systems development and IoT integration, and concluding with industrial automation and PLC programming. The primary deliverable was a complete Smart Home System incorporating RFID access control, sensor-based monitoring, cloud connectivity, and mobile application interfaces.

\noindent The internship achieved its core objectives of bridging academic theory with real-world application, while developing both technical competencies and professional skills. Key learning outcomes included proficiency in microcontroller programming, sensor integration, wireless communication protocols, and industrial control systems. Additionally, the program enhanced teamwork, technical communication, and project management capabilities within an industrial environment.

\noindent This report presents a comprehensive overview of the training activities, technical procedures, learning outcomes, and operational challenges encountered during the internship. The experience provided valuable insights into current industry practices and emerging technologies, while laying a strong foundation for career development in embedded systems and automation engineering. The internship successfully fulfilled academic requirements and delivered practical preparation for professional engineering practice.

\newpage
% Table of Contents
\tableofcontents
\vspace{40pt}

\vspace{0.2cm}
% List of Figures
\addcontentsline{toc}{chapter}{List of Figures}
\listoffigures
\vspace{-10pt}

\newpage

% Abbreviations
{\fontsize{14}{16.8}\selectfont\bfseries\centering ABBREVIATIONS\par}
\vspace{10pt}
\addcontentsline{toc}{chapter}{Abbreviations}

\begin{tabular}{ll}
AC & Alternating Current \\
API & Application Programming Interface \\
AT & Attention (command set) \\
BCD & Binary Coded Decimal \\
DC & Direct Current \\
EARSO & East African Research Services Organization \\
GSM & Global System for Mobile Communications \\
I2C & Inter-Integrated Circuit \\
IC & Integrated Circuit \\
IDE & Integrated Development Environment \\
IoT & Internet of Things \\
KIRDI & Kenya Industrial Research and Development Institute \\
LCD & Liquid Crystal Display \\
LED & Light Emitting Diode \\
MTIC & Ministry of Trade, Industry, and Cooperatives \\
PLC & Programmable Logic Controller \\
PPT & PowerPoint \\
PWM & Pulse Width Modulation \\
RC & Resistor-Capacitor \\
RFID & Radio Frequency Identification \\
SDK & Software Development Kit \\
SME & Small and Medium Enterprises \\
SPI & Serial Peripheral Interface \\
SR & Set-Reset \\
TIRDO & Tanzania Industrial Research and Development Organization \\
UCU & Uganda Christian University \\
UIRI & Uganda Industrial Research Institute \\
\end{tabular}

\newpage

% Nomenclature
{\fontsize{14}{16.8}\selectfont\bfseries\centering NOMENCLATURE\par}
\vspace{10pt}
\addcontentsline{toc}{chapter}{Nomenclature}

\begin{tabular}{ll}
7400 series & Logic gate integrated circuit family \\
Arduino UNO & Open-source microcontroller development board \\
ESP8266 & Low-cost Wi-Fi microcontroller chip \\
Expo CLI & Command-line interface for React Native development \\
Firebase & Google's mobile and web application development platform \\
HC-SR04 & Ultrasonic distance sensor module \\
MIFARE & Contactless smart card technology standard \\
NodeMCU & ESP8266-based development board \\
RC522 & RFID reader/writer module \\
React Native & Cross-platform mobile application development framework \\
STEP 7 & Siemens programming software for PLC systems \\
Tinkercad & Web-based circuit design and simulation platform \\
\end{tabular}

\newpage

\pagenumbering{arabic}
\fancyhf{}  % Clear all header/footer settings
\fancyfoot[R]{Page \thepage}  % Set right-aligned "Page X" for main content

% Redefine plain page style for chapter pages
  \fancypagestyle{plain}{%
  \fancyhf{}  % Clear all header/footer settings for plain style
  \renewcommand{\headrulewidth}{0pt}
  \fancyfoot[R]{Page \thepage}  % Set right-aligned "Page X" for chapter pages
} 

\chapter[CHAPTER ONE INTRODUCTION AND BACKGROUND TO UIRI]{CHAPTER ONE \\INTRODUCTION AND BACKGROUND TO UIRI}

\section{Introduction}
\noindent The internship program is a core requirement for undergraduate students in the Department of Computing and Technology at Uganda Christian University. It provides students with an opportunity to apply classroom theory in practical, real-world industrial settings. From 22\textsuperscript{nd} May to 18\textsuperscript{th} July 2025, I undertook an internship at the Uganda Industrial Research Institute (UIRI), a government parastatal dedicated to industrial and technological development in Uganda.

\noindent This chapter provides an overview of the hosting organization its background, mandate, activities, and the relevance of the internship. It also highlights the connection between academic training and industrial research and development at UIRI.

\section{Organizational Profile and Background}
\noindent The Uganda Industrial Research Institute (UIRI) traced its origins to the East African Research Services Organization (EARSO), which was established during the East African Federation of the 1970s to serve Uganda, Kenya, and Tanzania. After the federation dissolved in 1977, EARSO was succeeded by KIRDI (Kenya) and TIRDO (Tanzania), while Uganda lacked a national R\&D institute until 2002.

\noindent In 2002, UIRI was formally established by the Government of Uganda under the leadership of President Yoweri Kaguta Museveni. The institute was supported by a grant from the Chinese government, which contributed equipment and funding. Under the stewardship of Professor Charles Kwesiga, an industrial engineer, UIRI quickly emerged as one of Africa's premier research and development institutions.

\noindent UIRI operated from two main campuses its headquarters in Nakawa and the Machining Manufacturing Industrial Skills Development Centre (MMISDC) in Namanve. The Namanve facility began construction in 2018 and was fully commissioned on 15\textsuperscript{th} January 2020 by President Museveni.

\section{Nature and Role of UIRI}
\noindent UIRI was a government agency that was initially under the Ministry of Trade, Industry, and Cooperatives (MTIC) and now falls under the Ministry of Science, Technology, and Innovation. The institute served as a platform for value addition, technology incubation, applied research, and SME support.

\noindent Key services included:
\begin{itemize}
    \item Product and process design
    \item Fabrication of equipment and machinery
    \item Analytical laboratory services (chemistry and microbiology)
    \item Technical support to entrepreneurs and innovators
    \item Business incubation and skills development
\end{itemize}

\noindent UIRI operated as a computerized and technology-intensive organization that bridged the gap between academic research, industrial application, and national development goals.

\section{Vision, Mission, and Mandate}
\noindent\textbf{Vision:} To be a model institution and center of excellence for the incubation of industry, and a pioneer of self-financing research and development that elevates the level of technology in Uganda and the region.

\noindent\textbf{Mission:} To catalyze the socio-economic transformation of Uganda and the region through enhanced technology use.

\noindent\textbf{Mandate:} To engage in applied research and activities such as value addition that will result in the rapid industrialization of Uganda.

\newpage
\section{Objectives of UIRI}
\begin{itemize}
    \item To carry out applied research for product development and innovation.
    \item To develop or acquire appropriate technologies to support a competitive industrial sector.
    \item To promote value addition that transforms raw materials into marketable products.
    \item To bridge the gap between academia, government, and private sector through commercialization of research.
\end{itemize}

\section{Key Activities of UIRI}

\subsection{Value Addition and Product Development}
\noindent UIRI identified new uses for raw materials and developed value-added products through prototyping and innovation.

\subsection{Pilot Processing Plants}
\noindent These served as training and demonstration units for local entrepreneurs and facilitated small-scale product testing and development.

\subsection{Business Incubation}
\noindent The institute offered incubation services such as workspace, laboratory access, skills training, marketing, and business development support to startups and SMEs.

\subsection{Technology Transfer}
\noindent UIRI assisted entrepreneurs in acquiring and adapting appropriate technologies from within or outside the country, helping to improve production and competitiveness.

\subsection{Analytical Laboratory Services}
\noindent The institute provided chemical and microbiological testing services that supported product formulation, quality control, and market readiness.

\newpage
\section{Internship Scope and Relevance}
\noindent As a Computer Science student with an interest in embedded systems and automation, I was attached to the Embedded Systems Unit within the Engineering Department. The internship exposed me to practical technologies including:
\begin{itemize}
    \item Microcontroller programming using Arduino and ESP8266
    \item Circuit design and simulation using Proteus and Tinkercad
    \item Logic control systems using Siemens Step 7 and PLCs
    \item Interfacing sensors and actuators in home automation
    \item Collaborative project planning and documentation
\end{itemize}

\noindent The training provided essential hands-on experience in hardware design, programming, troubleshooting, and teamwork, preparing me for real-world challenges in the fields of IoT and digital control systems.

\section{Training Objectives}
\noindent The objectives of the internship at UIRI were to:
\begin{itemize}
    \item Gain practical exposure to electronics and embedded systems.
    \item Understand the industrial application of microcontrollers and control systems.
    \item Develop a functional embedded project from planning to deployment.
    \item Improve soft skills including teamwork, communication, and technical reporting.
\end{itemize}

\newpage
\chapter[CHAPTER TWO ACTIVITIES AND PROCEDURES]{CHAPTER TWO \\ACTIVITIES AND PROCEDURES}

\section{Introduction}

\noindent This chapter summarizes the key activities and procedures undertaken during the nine-week internship at Uganda Industrial Research Institute (UIRI). The internship was supervised by experienced UIRI personnel, each with specialized expertise: Mr. Atwine Philip (Digital Electronics), Mr. Adaon Michel (Analog Electronics), Mr. Ngobi Elijah and Mr. Mutesasira Bashir (Embedded Systems and IoT), Mr. Ainebyona Savior and Ms. Annah Faith (Industrial Automation and PLC Programming). They provided technical guidance and mentorship throughout the internship period. The following sections provide detailed descriptions of weekly activities, objectives, tools used, technical procedures, and key learning outcomes, verified under the appropriate supervisors as per the official log book.

\section{Week 1: Digital Electronics Foundations}

\noindent\textbf{Description:} The first week laid the groundwork for understanding digital electronics, focusing on combinational logic circuits and their practical applications. I engaged in both theoretical learning and hands-on experimentation to build a solid foundation in digital logic design. Activities included introduction to multiplexers, decoder circuits, diode biasing, and flip-flop design, culminating in a smart traffic light simulation.

\noindent\textbf{Objectives:}
\begin{itemize}
    \item Master combinational logic circuit design including 2:1 and 4:1 multiplexers, BCD-to-seven-segment decoders, and SR flip-flops.
    \item Achieve proficiency in Tinkercad simulation platform and UIRI digital experimental hardware.
    \item Develop systematic debugging techniques for identifying logic gate malfunctions and timing issues.
\end{itemize}

\newpage
\noindent\textbf{Tools and Materials Used:}
\begin{multicols}{2}
\begin{itemize}
    \item Tinkercad simulation platform
    \item UIRI digital experimental box
    \item Logic gate ICs (7400 series)
    \item Toggle switches
    \item LEDs
    \item Resistors
    \item Breadboard
    \item Wires
\end{itemize}
\end{multicols}

\noindent\textbf{Technical Procedure:}
\begin{itemize}
    \item Constructed and verified truth tables for 2:1 multiplexers, decoder circuits, and flip-flops using Tinkercad simulation.
    \item Assembled physical circuits on breadboards, validating logic operations through LED indicator patterns.
    \item Programmed a traffic light state machine using flip-flop sequential logic.
\end{itemize}

\noindent\textbf{Key Learning Outcomes:} Established foundational competency in digital logic design principles, gained fluency with simulation-to-hardware workflow, and acquired systematic approach to combinational circuit verification.

\begin{figure}[H]
\centering
% Top image (full width)
\includegraphics[width=0.97\textwidth]{img/4-input multiplexer simulation on Tinkercad.png}
\caption{4-input multiplexer circuit simulation using Tinkercad platform}
\label{fig:multiplexer-sim}

\vspace{3.5em}

% Bottom row: two images side by side
\begin{minipage}[t]{0.49\textwidth}
  \centering
  \includegraphics[width=1.0\linewidth]{img/7 segment Display using an IC.png}
  \caption{Seven-segment display implementation with decoder IC}
  \label{fig:7segment-display}
\end{minipage}
\hfill
\begin{minipage}[t]{0.49\textwidth}
  \centering
  \includegraphics[width=1.0\linewidth]{img/Experimenting with Digital Trainer box.png}
  \caption{Hands-on digital logic experiments using UIRI digital trainer box}
  \label{fig:digital-trainer}
\end{minipage}
\end{figure}


\newpage
\section{Week 2: Analog Electronics and Measurement Techniques}

\noindent\textbf{Description:} Week two introduced fundamental concepts of analog electronics, emphasizing component behavior and measurement techniques. I conducted practical experiments to understand electrical properties and signal analysis.

\noindent\textbf{Objectives:}
\begin{itemize}
    \item Analyze voltage-current characteristics of PN junction diodes under forward and reverse bias conditions.
    \item Characterize capacitor charging/discharging time constants and transistor switching thresholds.
    \item Master oscilloscope operation for AC signal measurement and function generator frequency synthesis.
\end{itemize}

\noindent\textbf{Tools and Materials Used:}
\begin{multicols}{2}
\begin{itemize}
    \item Function generator
    \item Oscilloscope
    \item Diodes
    \item Capacitors
    \item Transistors
    \item Resistors
    \item Analog experimental box
    \item Breadboard
    \item Wires
    \item DC power supply
\end{itemize}
\end{multicols}

\noindent\textbf{Technical Procedure:}
\begin{itemize}
    \item Investigated RC time constants through capacitor charge/discharge cycles with oscilloscope monitoring.
    \item Calibrated light-dependent resistor sensitivity curves across varying illumination levels.
\end{itemize}

\noindent\textbf{Key Learning Outcomes:} Developed precision in analog measurement techniques, gained understanding of semiconductor device physics, and established competency in laboratory instrumentation protocols.

\begin{figure}[H]
\centering
% Top row: two images side by side
\begin{minipage}[t]{0.49\textwidth}
  \centering
  \includegraphics[width=1.0\linewidth, height=7.5cm, keepaspectratio]{img/Using the Analog trainer box.png}
  \caption{Analog circuit component testing and characterization}
  \label{fig:analog-components}
\end{minipage}
\hfill
\begin{minipage}[t]{0.49\textwidth}
  \centering
  \includegraphics[width=1.0\linewidth, height=7.5cm, keepaspectratio]{img/Experimenting with Analog components.png}
  \caption{Practical analog circuit implementation on trainer platform}
  \label{fig:analog-trainer}
\end{minipage}

\vspace{4em}

% Bottom image: spans most of the text width
\begin{minipage}[t]{0.98\textwidth}
  \centering
  \includegraphics[width=0.8\linewidth]{img/Using the Oscilloscpe and functional generator.png}
  \caption{Signal analysis using oscilloscope and function generator setup}
  \label{fig:oscilloscope-setup}
\end{minipage}
\end{figure}

\newpage
\section{Weeks 3--7: Embedded Systems and IoT Development}

\subsection{Embedded Systems Development (Weeks 3--4)}

\noindent\textbf{Description:} This period focused on Arduino-based microcontroller programming, sensor integration, and modular system design. I worked on an intruder detection system and initial smart home automation modules, gaining hands-on experience in hardware assembly and debugging.

\noindent\textbf{Objectives:}
\begin{itemize}
    \item Implement ultrasonic sensor distance measurement algorithms with sub-centimeter \\accuracy.
    \item Engineer modular code architecture for scalable home security system expansion.
    \item Resolve LCD display voltage regulation and I2C communication protocol errors.
\end{itemize}

\noindent\textbf{Tools and Materials Used:}
\begin{multicols}{2}
\begin{itemize}
    \item Arduino UNO
    \item ESP8266 Wi-Fi module
    \item HC-SR04 ultrasonic sensor
    \item Buzzer
    \item LCD display
    \item Arduino IDE
    \item Breadboards
    \item Jumper wires
\end{itemize}
\end{multicols}

\noindent\textbf{Technical Procedure:}
\begin{itemize}
    \item Calibrated HC-SR04 sensor timing parameters for precise distance calculations using pulse-width measurement.
    \item Integrated multi-tone buzzer alarm sequences triggered by configurable proximity thresholds.
    \item Diagnosed and corrected LCD backlight voltage issues through systematic power supply analysis.
    \item Optimized sensor polling rates to balance responsiveness with power consumption.
\end{itemize}

\noindent\textbf{Key Learning Outcomes:} Acquired expertise in sensor signal processing, developed systematic hardware debugging methodology, and established foundation for real-time embedded system design.

\begin{figure}[H]
\centering
\includegraphics[width=0.65\textwidth]{img/RGB control with IR remote and Door bell module.jpg}
\caption{Integrated RGB lighting control and doorbell notification system}
\label{fig:rgb-doorbell}
\end{figure}

\subsection{Smart Home System Implementation (Weeks 5--6)}

\noindent\textbf{Description:} Focused on integrating security features such as RFID access control, wireless module testing, and assembling the full smart home system. Cloud connectivity was established using Firebase, and a mobile application was developed for remote monitoring and control.

\noindent\textbf{Objectives:}
\begin{itemize}
    \item Configure RC522 RFID authentication with encrypted card ID verification and access logging.
    \item Establish bidirectional Firebase Realtime Database communication for IoT device coordination.
    \item Deploy React Native mobile application with real-time system status monitoring and control interfaces.
\end{itemize}

\noindent\textbf{Tools and Materials Used:}
\begin{multicols}{2}
\begin{itemize}
    \item Arduino UNO
    \item RC522 RFID module
    \item GSM module
    \item ESP8266 NodeMCU
    \item Firebase SDK
    \item React Native
    \item Expo CLI
    \item Mobile devices
\end{itemize}
\end{multicols}

\noindent\textbf{Technical Procedure:}
\begin{itemize}
    \item Programmed MIFARE card reading protocols with SPI communication interface configuration.
    \item Implemented GSM AT command sequences for SMS notification backup systems.
    \item Structured Firebase database schema with separate command and status node hierarchies.
    \item Developed React Native components with Firebase authentication and real-time data binding.
\end{itemize}

\noindent\textbf{Key Learning Outcomes:} Mastered RFID security protocol implementation, gained proficiency in cloud-based IoT architecture design, and acquired mobile application development skills for hardware control.

\vspace{1.2cm}
\begin{figure}[H]
\centering
\begin{minipage}[t]{0.49\textwidth}
  \centering
  \includegraphics[width=1.0\linewidth]{img/RFID module with relay switch opened using CID.jpg}
  \caption{RFID-based access control system with relay activation}
  \label{fig:rfid-system}
\end{minipage}
\hfill
\begin{minipage}[t]{0.49\textwidth}
  \centering
  \includegraphics[width=1.0\linewidth]{img/full home monitoring system on dual boards.jpg}
  \caption{Complete smart home monitoring system with Arduino and NodeMCU integration}
  \label{fig:full-system}
\end{minipage}
\end{figure}

\newpage
\subsection{System Optimization and Real-Time Control (Week 7)}

\noindent\textbf{Description:} This week addressed system troubleshooting, authentication implementation, and real-time control optimization. I resolved hardware-software integration challenges, mitigated timer conflicts, and refined communication protocols for stable operation.

\noindent\textbf{Objectives:}
\begin{itemize}
    \item Implement Firebase user credential embedding for secure cloud authentication protocols.
    \item Eliminate hardware timer conflicts between IRremote library and buzzer PWM generation.
    \item Optimize Firebase database polling intervals for sub-second command response latency.
\end{itemize}

\noindent\textbf{Tools and Materials Used:}
\begin{multicols}{2}
\begin{itemize}
    \item NodeMCU ESP8266
    \item Arduino UNO
    \item toneAC library
    \item IRremote library
    \item Firebase Realtime Database
    \item React Native app
\end{itemize}
\end{multicols}

\noindent\textbf{Technical Procedure:}
\begin{itemize}
    \item Hardcoded Firebase authentication tokens with encrypted credential storage for autonomous operation.
    \item Reassigned hardware timer resources and adopted non-blocking toneAC library for conflict resolution.
    \item Implemented Firebase database listeners with exponential backoff retry mechanisms.
    \item Benchmarked system response times under varying network latency conditions.
\end{itemize}

\noindent\textbf{Key Learning Outcomes:} Developed advanced system integration problem-solving skills, gained expertise in embedded authentication security, and mastered real-time IoT communication optimization techniques.

\section{Weeks 8--9: Industrial Automation and PLC Programming}

\noindent\textbf{Description:} The final weeks focused on industrial automation concepts and Programmable Logic Controller (PLC) programming using Siemens STEP 7 software. I engaged in ladder logic design, timer and counter functions, and complex automation scenarios including traffic light control and conveyor sorting systems. The internship culminated with project presentations and professional development sessions.

\noindent\textbf{Objectives:}
\begin{itemize}
    \item Program Siemens S7-300 ladder logic for industrial conveyor sorting with multi-sensor decision trees.
    \item Design timer-based traffic light control systems with pedestrian crossing integration and emergency override.
    \item Execute comprehensive project presentations demonstrating technical competency and system integration mastery.
\end{itemize}

\noindent\textbf{Tools and Materials Used:}
\begin{multicols}{2}
\begin{itemize}
    \item Siemens STEP 7 software
    \item Codesys software
    \item PLC training boards
    \item Ladder logic programming tools
    \item Industrial sensors and actuators
    \item Presentation software
\end{itemize}
\end{multicols}

\noindent\textbf{Technical Procedure:}
\begin{itemize}
    \item Analyzed PLC memory architecture including input/output modules, data blocks, and function block organization.
    \item Developed ladder logic sequences for material handling automation with proximity sensor feedback.
    \item Simulated industrial processes using STEP 7 hardware configuration and program debugging tools.
    \item Compiled technical documentation including system specifications, wiring diagrams, and operational procedures.
    \item Presented integrated smart home system to evaluation panel including technical demonstration and Q\&A session.
\end{itemize}

\noindent\textbf{Key Learning Outcomes:} Achieved proficiency in industrial automation programming standards, mastered PLC system design methodology, and developed professional technical presentation and documentation skills.

\begin{figure}[H]
\centering
\includegraphics[width=0.8\textwidth]{img/Simulation of Box lifter program on Simens Step 7.jpg}
\caption{Material handling automation sequence programmed in Siemens STEP 7 environment}
\label{fig:step7-simulation}
\end{figure}

\begin{figure}[H]
\centering
\includegraphics[width=0.65\textwidth]{img/Tour via the industry to see PLCs in action.jpg}
\caption{Industrial PLC systems observation during facility tour}
\label{fig:plc-tour}
\end{figure}

\section{Week 9: Project Presentation and Closing Remarks}

\noindent\textbf{Objective:} Execute professional project presentations demonstrating comprehensive technical integration and system functionality.

\noindent\textbf{Tools and Materials Used:}
\begin{itemize}
    \item Final project hardware setup
    \item Presentation slides (PPT)
\end{itemize}

\noindent\textbf{Technical Procedure:}
\begin{itemize}
    \item Demonstrated complete smart home system functionality including RFID authentication, motion detection, automated fan control, and Firebase cloud synchronization.
    \item Presented technical architecture, implementation challenges, and solution methodologies to UCU and UIRI leadership panel.
    \item Engaged in technical discussion covering system scalability, security considerations, and potential commercial applications.
\end{itemize}

\noindent\textbf{Key Learning Outcomes:} Developed confidence in technical communication under professional evaluation, refined ability to articulate complex system integration concepts, and received recognition for collaborative teamwork and technical innovation.

\begin{figure}[H]
\centering
\includegraphics[width=0.7\textwidth]{img/Engineer David Ilukol giving closing remarks.png}
\caption{Closing ceremony remarks by Engineer David Ilukol}
\label{fig:closing-remarks}
\end{figure}

\newpage
\chapter[CHAPTER THREE LEARNING OUTCOMES AND REFLECTIONS]{CHAPTER THREE \\LEARNING OUTCOMES \& REFLECTIONS}

\section{Introduction}

\noindent This chapter presents a comprehensive reflection on the technical competencies and professional skills acquired during the UIRI internship experience. The structured progression through multiple engineering disciplines provided both depth and breadth in practical technology application.

\section{Technical Lessons Learned}

\subsection{Digital Logic Design and Circuit Implementation}
\noindent Mastered systematic approach to combinational and sequential logic design, including multiplexer selection logic, decoder truth table verification, and flip-flop state machine programming. Gained proficiency in simulation to hardware validation workflows using Tinkercad and physical breadboard implementations.

\subsection{Embedded Microcontroller Programming and Sensor Integration}
\noindent Developed expertise in Arduino IDE programming with focus on real-time sensor data processing, interrupt-driven event handling, and modular code architecture. Acquired specialized skills in ultrasonic sensor calibration, RFID authentication protocols, and multi-peripheral communication management.

\subsection{Cloud-Based IoT Architecture and Mobile Application Development}
\noindent Established competency in Firebase Realtime Database configuration, bidirectional data synchronization, and authentication security implementation. Gained proficiency in React Native mobile development with emphasis on hardware control interfaces and real-time status monitoring.

\newpage
\subsection{Industrial Automation and PLC System Programming}
\noindent Achieved fluency in Siemens STEP 7 ladder logic programming, including timer functions, counter operations, and sensor-based decision logic. Developed understanding of industrial automation standards, safety protocols, and system scalability considerations.

\section{Advanced Problem-Solving and Engineering Analysis}

\noindent The internship emphasized systematic troubleshooting methodologies, requiring analysis of hardware-software integration challenges, timing conflicts, and communication protocol optimization. This experience developed critical thinking skills essential for complex system debugging and performance optimization.

\section{Professional Development and Communication Skills}

\subsection{Technical Collaboration and Knowledge Transfer}
\noindent Enhanced ability to work within multidisciplinary teams, and communicate complex technical concepts to both technical and non-technical audiences. Developed skills in technical mentorship and knowledge sharing.

\subsection{Professional Documentation and Presentation}
\noindent Improved technical writing capabilities through system documentation, code commenting, and report preparation. Strengthened presentation skills through formal project demonstrations.

\subsection{Project Management and Resource Optimization}
\noindent Developed time management skills for concurrent project development, resource allocation for shared laboratory equipment, and adaptive planning for technical challenges and changing requirements.

\section{Career Development and Industry Insights}

\noindent The comprehensive exposure to embedded systems, IoT architecture, and industrial automation clarified career interests and provided practical understanding of current industry technologies and methodologies. The mentorship experience highlighted the importance of continuous learning and professional development in rapidly evolving technical fields.

\newpage
\chapter[CHAPTER FOUR CHALLENGES, RECOMMENDATIONS AND CONCLUSION]{CHAPTER FOUR \\CHALLENGES, RECOMMENDATIONS AND CONCLUSION}
\vspace{-0.4cm}
\section{Introduction}
\vspace{-0.2cm}
\noindent This chapter analyzes the primary challenges encountered during the UIRI internship and provides evidence-based recommendations for program enhancement. The assessment is based on direct experience with training resources, technical tools, and learning methodologies.

\section{Challenges Faced}

\subsection{Internet Connectivity Limitations}
\noindent Inadequate internet connectivity at the UIRI institute significantly impacted cloud-based development activities, particularly Firebase database synchronization and mobile application testing phases.

\subsection{Compressed Development Timeline for Complex Integration}
\noindent The intensive nine-week schedule, while comprehensive, provided limited time for iterative optimization of embedded systems projects and advanced topics such as PCB design and fabrication experience.

\subsection{Siemens STEP 7 Software Configuration Issues}
\noindent The Siemens STEP 7 programming environment presented significant operational challenges on UIRI's computer systems, preventing full utilization of available PLC training boards and limiting hands-on industrial automation experience.

\subsection{Limited Time for Advanced Manufacturing Processes}
\noindent The program lacked sufficient time allocation for PCB design and fabrication experience, which would have provided valuable insight into complete product development cycles.

\subsection{Software-Hardware Configuration Dependencies}
\noindent Embedded module initialization, particularly GSM AT command configuration and Firebase SDK integration, required extensive setup procedures that occasionally disrupted development momentum.

\section{Recommendations}

\subsection{Enhanced Internet Infrastructure}
\noindent UIRI should invest in reliable, high-speed internet connectivity with backup systems to ensure consistent cloud-based development capabilities and uninterrupted access to online technical resources.

\subsection{Extended Program Duration with PCB Design Module}
\noindent Consider extending the internship program to 10 weeks to include dedicated time for PCB design and fabrication experience, allowing trainees to complete full product development cycles from concept to manufactured prototype.

\subsection{Licensed Software Procurement for PLC Training}
\noindent Acquire proper licensed versions of Siemens STEP 7 software and ensure compatibility with training computer systems to enable full utilization of PLC training boards and provide comprehensive industrial automation experience.

\subsection{Alumni Mentorship Program Implementation}
\noindent Establish an alumni mentorship program involving past successful trainees in guidance roles to help incoming students transition more smoothly and offer valuable peer-led insights into navigating the training program effectively.

\subsection{Extended Project Development Timeline}
\noindent Allocate additional time for embedded systems and IoT development phases to allow for comprehensive system optimization, thorough testing, and detailed documentation of complex projects.

\newpage

\section{Conclusion}

\noindent The nine-week UIRI internship successfully bridged academic theory with practical industry application through structured progression from foundational electronics to advanced embedded systems and industrial automation.

\noindent The program delivered comprehensive technical competencies across digital circuit design, analog measurement, embedded programming, IoT integration, and PLC automation. Expert mentorship from UIRI technical staff ensured deep understanding of complex technologies through hands-on methodology.

\noindent Challenging project phases developed critical engineering judgment and systematic troubleshooting capabilities. Resolving hardware timer conflicts and optimizing cloud communication protocols provided authentic professional experience beyond traditional academic settings.

\noindent Professional development was equally significant, with substantial improvement in technical communication, collaborative problem-solving, and project management. Presenting to professional evaluation panels and maintaining comprehensive documentation provided essential career preparation.

\noindent The internship clarified professional interests while providing realistic perspectives on engineering practice. Exposure to current industry technologies and experienced practitioner mentorship offered valuable career development insights.

\noindent Operational challenges, including equipment limitations and infrastructure constraints, provided additional learning opportunities in resource management and adaptive problem-solving that contributed to professional growth.

\noindent The internship achieved its primary objectives of developing practical engineering competencies and providing career guidance. The technical skills and professional insights gained provided a solid foundation for advanced academic pursuits and successful career development in embedded systems, IoT, and industrial automation.

\newpage
% References
{\fontsize{14}{16.8}\selectfont\bfseries\centering REFERENCES\par}
\vspace{10pt}
\addcontentsline{toc}{chapter}{References}

\noindent\hangindent=1.4em [1] \url{https://www.uiri.go.ug}

\noindent\hangindent=1.4em [2] \url{https://www.uiri.go.ug/about}

\noindent\hangindent=1.4em [3] Arduino component tutorials and projects: 
\url{https://projecthub.arduino.cc}

\noindent\hangindent=1.4em [4] Circuit design and simulation platform: 
\url{https://www.tinkercad.com}

\noindent\hangindent=1.4em [5] More on the Home monitoring system: \\
\url{https://drive.google.com/drive/folders/1kpCOZwo20NtfggT5KUVm3Gei5rfW0cfG?usp=drive_link}

\noindent\hangindent=1.4em [6] More images can be got from this link: \\
\url{https://drive.google.com/drive/folders/13eKcRinhntNd0XVgoEnWzphzhrrDpTjw?usp=sharing}

\noindent\hangindent=1.4em [7] Al Tareq, Abdulla, Md Riad Mostofa, Md Juel Rana, and Md Sadiqur Rahman. (2024). "A comprehensive review of intelligent home automation systems using embedded devices and IoT." \textit{Control Systems and Optimization Letters} 2, no. 2: 198-203.
\vspace{0.2cm}

\noindent\hangindent=1.4em [8] Kimutai, Metto S., Omieno K. Kelvin, and Ondulo M. Jasper. (2022). "Challenges and Opportunities for Smart Homes Deployment in Developing Countries: A Case Study of the User Perspective in Kenya." \textit{Open Access Library Journal} 9, no. 7: 1-8.
\vspace{0.2cm}


\newpage
\appendix

% Appendix
{\fontsize{14}{16.8}\selectfont\bfseries\centering APPENDIX\par}
\vspace{10pt}
\addcontentsline{toc}{chapter}{Appendix}

\begin{figure}[H]
\centering
\includegraphics[width=0.55\textwidth]{img/The Executive Director and the HR team.png}
\caption{UIRI Executive Director and Human Resources team during orientation program}
\label{fig:uiri-leadership}
\end{figure}

\begin{figure}[H]
\centering
\includegraphics[width=0.47\textwidth]{img/Powering the components using an Analog box.png}
\caption{Component power supply configuration using analog trainer box}
\label{fig:analog-power}
\end{figure}

\begin{figure}[H]
\centering
\includegraphics[width=0.44\textwidth]{img/Using a multimeter to mearsure voltage.png}
\caption{Voltage measurement techniques using digital multimeter}
\label{fig:multimeter-usage}
\end{figure}

% \begin{figure}[H]
% \centering
% \includegraphics[width=0.67\textwidth]{img/part of the current smart home app interface.png}
% \caption{Smart home mobile application interface showing real-time control and monitoring capabilities}
% \label{fig:mobile-app}
% \end{figure}

\end{document}