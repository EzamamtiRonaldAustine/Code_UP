\documentclass[12pt,a4paper]{report}
\usepackage{geometry}
\usepackage{hyperref}
\hypersetup{
  colorlinks=true,
  linkcolor=black,
  citecolor=black,
  filecolor=black,
  urlcolor=black
}
\usepackage{graphicx}
\usepackage{booktabs}
\usepackage{longtable}
\usepackage{array}
\usepackage{caption}
\usepackage{float}
\usepackage{amsmath}
\usepackage{cite}
\usepackage[table]{xcolor}
\usepackage{setspace}
\usepackage{fontspec}
\usepackage{etoolbox}
\usepackage{titlesec}

\geometry{margin=1in}
\setmainfont{Trebuchet MS}
\onehalfspacing
\renewcommand{\contentsname}{Table of Contents}

\titleformat{\chapter}[display]
  {\normalfont\huge\bfseries}{}{0pt}{\centering}
\titlespacing{\chapter}{0pt}{-40pt}{10pt}

\makeatletter
\patchcmd{\@makechapterhead}{\vspace*{50\p@}}{\vspace*{0pt}}{}{}
\patchcmd{\@makeschapterhead}{\vspace*{50\p@}}{\vspace*{0pt}}{}{}
\makeatother
\titlespacing\section{0pt}{5pt}{5pt}

\begin{document}

% ---------------- Title Page ---------------- %
\begin{titlepage}
    \centering
    \includegraphics[width=0.9\textwidth]{img/UCU.png}\\
    \vspace*{2cm}
    {\Huge\bfseries Smart Fish Pond Management System: An IoT-Based Approach for Sustainable Aquaculture in Uganda \par}
    \vspace{2cm}
    {\LARGE\itshape\textbf Project Proposal\par}
    \vspace{2cm}
    {\Large\textbf by:\par}
    \vspace{1cm}
    {\Large\textbf EZAMAMTI RONALD AUSTINE \par}
    \vspace{0.5cm}
    {\large S23B23-018\par}
    \vspace{0.5cm}
    {\large B24252\par}
    \vspace{1.5cm}
    {\large \today\par}
\end{titlepage}

% ---------------- Executive Summary ---------------- %
\begin{center}
    {\LARGE\textbf Executive Summary}
\end{center}
\noindent
fish export revenues declined by 21.9\% in 2024, from UGX 514.5 billion to UGX 395.7 billion, with export volumes falling by 27.8\%~\cite{bankuganda2025}. This decline, largely attributed to overfishing, illegal practices, and water pollution, highlights the need for sustainable aquaculture. Tilapia and catfish, the main pond-farmed species, have potential to fill this gap but face persistent challenges such as poor water quality management, high mortality, and lack of affordable monitoring systems~\cite{tumwesigye2022effect, byabasaija2025unlocking}.

This proposal presents the Smart Fish Pond Management System: an IoT-enabled embedded solution that continuously monitors dissolved oxygen, pH, turbidity, temperature, ammonia, and water levels, while automating corrective actions like aeration and pumping. Farmers will receive real-time alerts through GSM and mobile applications. By leveraging affordable embedded hardware and solar energy, the system aims to improve survival rates, enhance yields, and stabilize fish farming incomes.  

The system directly contributes to SDGs 2, 6, 9, and 13, aligns with Uganda’s NDP IV (2025–2030), and supports Vision 2040’s call for ICT-driven agro-industrialization. It will be implemented as a low-cost prototype using an ESP32 microcontroller, sensors, solar energy, and GSM connectivity, with future scalability to integrate AI-based computer vision for fish health monitoring.  

\tableofcontents
\clearpage

% ---------------- Chapter 1 ---------------- %
\chapter{Chapter One: Introduction and Background}
\noindent
alternative~\cite{aanyu2020potential}.

Despite its potential, aquaculture is constrained by persistent challenges. Farmers often lack knowledge of optimal water quality management, leading to fish mortality, reduced yields, and abandonment of ponds~\cite{tumwesigye2022effect}. Reports between 2020 and 2025 repeatedly highlight fish kills linked to low dissolved oxygen, pollution, and disease outbreaks. With Uganda’s Vision 2040 emphasizing modernized agriculture through ICT and the NDP IV prioritizing agro-industrialization, digital innovations in aquaculture are essential.  

\section{Problem Statement}
diseases~\cite{tumwesigye2022effect}.

support~\cite{byabasaija2025unlocking}.

\section{Objectives}
\subsection*{General Objective}
To design, prototype, and test an IoT-based Smart Fish Pond Management System for sustainable tilapia and catfish aquaculture in Uganda.  

\subsection*{Specific Objectives}
\begin{itemize}
    \item To design a real-time monitoring system for key water quality parameters (DO, pH, turbidity, ammonia, temperature, and water level).
    \item To automate corrective actions such as aeration, water pumping, and dosing based on threshold values.
    \item To integrate GSM and mobile platforms for real-time farmer alerts and decision-making.
    \item To test and evaluate the prototype under controlled pond environments.
    \item To assess the economic feasibility of the system for smallholder aquaculture farmers.
\end{itemize}

\section{Justification}
This project is justified on technical, economic, and policy grounds:
\begin{itemize}
inspection~\cite{prapti2022internet}.
markets~\cite{byabasaija2025unlocking}.
    \item \textbf{Policy:} Aligns with Uganda’s Vision 2040 (ICT in agriculture) and NDP IV (agro-industrialization).
    \item \textbf{Sustainability:} Solar-powered design supports climate resilience and SDG 13.
\end{itemize}

% ---------------- Chapter 2 ---------------- %
\chapter{Chapter Two: Literature Review}
\section{Current Developments in Precision Agriculture}
Global aquaculture increasingly uses IoT, edge computing, and AI. Jomsri et al.~\cite{jomsri2024prototype} developed a solar-powered water quality system integrating pH, DO, turbidity, and temperature sensors, showing effective real-time monitoring. Wang et al.~\cite{wang2021intelligent} described intelligent fish farms using AI to optimize feeding and water management. Prapti et al.~\cite{prapti2022internet} reviewed IoT applications in aquaculture, noting benefits such as mortality reduction but also challenges of affordability and scalability.  

\section{African Context}
In Sub-Saharan Africa, aquaculture is expanding but faces limitations. Clough et al.~\cite{clough2020innovative} tested a recirculating aquaculture system (RAS) in Kenya powered by solar PV, reducing costs and improving hatchery efficiency. Mramba and Kahindi~\cite{mramba2023pond} linked water quality in small-scale ponds to fish yields and diseases in arid regions, underscoring the importance of continuous monitoring. These examples highlight both potential and barriers for technology adoption in resource-limited settings.  

\section{Ugandan Context}
Studies in Uganda emphasize that pond aquaculture productivity is directly constrained by poor water quality. Tumwesigye et al.~\cite{tumwesigye2022effect} documented how ammonia, iron, and low DO impair tilapia and catfish productivity. Byabasaija et al.~\cite{byabasaija2025unlocking} assessed small-scale aquaculture in the Lake Victoria Basin and concluded that while viable, productivity is undermined by poor pond management and limited monitoring tools. Aanyu et al.~\cite{aanyu2020potential} noted that farmer organizations can enhance commercialization, but without supportive technologies, growth remains slow.  

\section{Identified Gaps in Research}
\begin{itemize}
    \item Few IoT prototypes tested in Ugandan aquaculture.
    \item Lack of affordable and scalable pond-specific solutions.
    \item Limited integration of mobile money and ICT platforms in aquaculture services.
\end{itemize}

% ---------------- Chapter 3 ---------------- %
\chapter{Chapter Three: Methodology}
\section{Methodology of the Proposed System}
The methodology adopts a co-design approach:
\begin{enumerate}
    \item Requirement analysis with tilapia and catfish farmers.
    \item Hardware prototyping using ESP32, GSM, and sensors.
    \item Firmware development in Arduino IDE for data acquisition and threshold-based automation.
    \item Field testing under controlled pond environments.
    \item Evaluation based on mortality reduction, system accuracy, and cost-effectiveness.
\end{enumerate}

\section{Hardware Design}
\begin{itemize}
    \item ESP32 (dual-core, Wi-Fi/BLE, low power).
    \item Sensors: DO, pH, DS18B20 temperature, turbidity, ammonia/nitrate, and water level.
    \item Actuators: relay-controlled aerators, pumps, and solenoid valves.
    \item GSM (SIM800L) for SMS alerts.
    \item Solar power: 12V panel, charge controller, and battery backup.
\end{itemize}

\section{Software Implementation}
Firmware will be developed on Arduino IDE with:
\begin{itemize}
    \item Sensor data acquisition and calibration.
    \item Threshold detection and actuator control.
    \item SMS-based alert system.
    \item Optional mobile/cloud dashboard for data visualization.
\end{itemize}

\section{Testing and Evaluation}
The system will be evaluated on:
\begin{itemize}
    \item Sensor accuracy against laboratory kits.
    \item Reduction in fish mortality rates compared to control ponds.
    \item Usability and affordability feedback from farmers.
    \item Solar power efficiency and uptime performance.
\end{itemize}

\section{Key Components}
The prototype will be built with essential embedded hardware and sensors, summarized in Table~\ref{tab:components}.

\begin{table}[H]
\centering
\renewcommand{\arraystretch}{1.3}
\begin{tabular}{|l|r|}
\hline
\rowcolor{gray!25}
\textbf{Component} & \textbf{Purpose} \\
\hline
ESP32 MCU & Main controller + IoT link \\
\hline
Dissolved Oxygen Sensor & Measures oxygen levels \\
\hline
pH Sensor & Tracks acidity/alkalinity \\
\hline
Turbidity Sensor & Detects water cloudiness \\
\hline
DS18B20 Temperature Sensor & Monitors pond temperature \\
\hline
Ammonia/Nitrate Sensor & Detects toxic waste buildup \\
\hline
Water Level Sensor & Monitors pond water levels \\
\hline
Relay Module & Switches aerator/pump \\
\hline
Aerator & Maintains dissolved oxygen \\
\hline
Water Pump & Circulates pond water \\
\hline
GSM Module & Sends SMS/alerts to farmer \\
\hline
Solar Panel + Battery & Provides off-grid power \\
\hline
LCD/OLED Display & Shows real-time data \\
\hline
Buzzer/LED Indicators & Local alerts \\
\hline
\end{tabular}
\caption{Core components and their purposes in the Smart Fish Pond Management System.}
\label{tab:components}
\end{table}


\section{Workplan (Four Weeks)}
\begin{table}[H]
\centering
\renewcommand{\arraystretch}{1.3}
\begin{tabular}{|c|p{11cm}|}
\hline
\textbf{Week} & \textbf{Activities} \\
\hline
1 & Requirement gathering and stakeholder consultation; procurement of ESP32, sensors, GSM module, solar components. \\
\hline
2 & Hardware assembly and sensor calibration; construction of pond simulation setup. \\
\hline
3 & Firmware programming; integration of GSM alerts; calibration of DO and pH sensors; testing power systems. \\
\hline
4 & Field deployment in selected ponds; monitoring and data collection; evaluation and final reporting. \\
\hline
\end{tabular}
\caption{Workplan for Smart Fish Pond Management System}
\end{table}

% ---------------- References ---------------- %
\renewcommand{\bibname}{References}
\addcontentsline{toc}{chapter}{References}
\bibliographystyle{IEEEtran}
\bibliography{SmartFish_References}

% ---------------- Appendix ---------------- %
\appendix
\chapter*{Appendix}
\addcontentsline{toc}{chapter}{Appendix}

\begin{figure}[H]
    \centering
    \includegraphics[width=0.8\textwidth]{img/esp32.png}
    \caption{ESP32 Microcontroller (Source: Arduino Store)}
\end{figure}

\begin{figure}[H]
    \centering
    \includegraphics[width=0.8\textwidth]{img/do_sensor.png}
    \caption{Dissolved Oxygen Sensor (Source: DFRobot)}
\end{figure}

\end{document}
