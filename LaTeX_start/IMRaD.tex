\documentclass[12pt,a4paper]{article}

% Packages for professional formatting
\usepackage{booktabs}   % for nice tables
\usepackage{array}      % for column formatting
\usepackage{geometry}   % adjust margins
\usepackage{setspace}   % line spacing
\usepackage{helvet}     % Helvetica font (clean scientific look)
\usepackage{longtable}  % allows tables to break across pages
\renewcommand{\familydefault}{\sfdefault} % set default font to sans-serif

\geometry{left=1in, right=1in, top=1in, bottom=1in}
\setstretch{1.3} % line spacing

\begin{document}

\section*{Peer‐Reviewed Embedded Pedagogy (PREP):
A New Approach to Promote and 
Improve Active Learning in Science Education}

\noindent
\textbf{Abstract.} \\
Active learning is proven to enhance student engagement and achievement in STEM, yet adoption of Open Educational Resources (OERs) remains limited due to faculty time constraints and lack of incentives. Orr and Kirkegaard (2022) propose \textit{Peer-Reviewed Embedded Pedagogy (PREP)} as a novel approach that integrates teaching resources directly into the research process. Under PREP, laboratories publish OERs prior to their associated research articles, embedding links within manuscripts to provide immediate educational value. Using a case study in vertebrate scavenger ecology, the authors demonstrate how PREP addresses barriers to active learning by offering four advantages for researchers—time efficiency, increased visibility, broader societal impacts, and graduate recruitment—and three advantages for students, including authentic exposure to data, visual context, and insight into the research process. PREP directly supports the \textit{AAAS Vision and Change} benchmarks by encouraging hypothesis-driven inquiry, fostering communities of scholars, and highlighting the passion behind science. While empirical validation is still needed, PREP presents a promising framework to simultaneously elevate teaching quality and research impact, offering a sustainable model for advancing active learning in undergraduate science education.

\vspace{1.5em}

\newpage
\section*{IMRaD Structure Analysis}

\textbf{Introduction:}  
The paper situates itself within the challenge of low OER adoption despite strong evidence that active learning improves STEM education. Barriers include time demands and lack of incentives in research-focused institutions. The authors propose PREP—embedding OERs within research outputs—as a solution to bridge research and teaching.

\medskip
\textbf{Methods:}  
Instead of empirical experimentation, this is a conceptual and pedagogical framework. PREP is explained as a workflow where OERs are created and peer-reviewed before the related scientific paper is published. The authors provide a case study (Orr 2019 scavenger ecology) to illustrate feasibility.

\medskip
\textbf{Results:}  
The authors identify four benefits for researchers (reduced time costs, improved visibility and citations, broader societal impact, and graduate recruitment) and three benefits for students (authentic engagement with primary research, visual and contextual support, and insights into the research process). PREP strengthens alignment between research and classroom practice.

\medskip
\textbf{Discussion:}  
The paper argues that PREP lowers barriers to OER adoption, improves teaching quality, and addresses eight \textit{AAAS Vision and Change} benchmarks. It explicitly links research to pedagogy and highlights potential long-term cultural shifts in academic science. Limitations are acknowledged: empirical tests of PREP are still pending, requiring replication and wider adoption. Recommendations include embedding OER links in supplementary materials for accessibility.
\newpage
\vspace{1.5em}

\renewcommand{\arraystretch}{1.3} % row spacing
\small % slightly smaller font for table

\begin{longtable}{>{\bfseries}p{4cm} p{11cm}}
\toprule
IMRaD Section & Supporting Evidence from Orr \& Kirkegaard (2022) \\
\midrule
\endfirsthead
\toprule
IMRaD Section & Supporting Evidence from Orr \& Kirkegaard (2022) \\
\midrule
\endhead
\midrule
\multicolumn{2}{r}{\textit{Continued on next page}} \\
\midrule
\endfoot
\bottomrule
\endlastfoot

Introduction 
& ``Active learning improves student performance... yet OER adoption remains low due to time demands and institutional barriers.'' \newline
``We propose that active learning exercises should, when feasible, be created and published before the research...'' \\

\addlinespace
Methods 
& ``PREP reverses the order: an OER is created and peer-reviewed first, then embedded into the scientific article.'' \newline
Case study: Orr (2019) on scavenger guild ecology linked directly to a PREP OER. \\

\addlinespace
Results 
& Four researcher benefits: (1) reduced time cost, (2) increased visibility and citations, (3) broader impacts, (4) graduate recruitment. \newline
Three student benefits: (1) authentic data engagement, (2) contextual visuals, (3) behind-the-scenes research insight. \\

\addlinespace
Discussion 
& ``PREP addresses eight AAAS benchmarks by aligning research with teaching, encouraging active participation, and fostering a community of scholars.'' \newline
``Limitations: empirical tests of PREP are not yet available and replication will be required.'' \\
\end{longtable}

\end{document}
