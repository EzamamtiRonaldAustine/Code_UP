\documentclass{beamer}

% Theme choice
\usetheme{Madrid}

% Package for better formatting
\usepackage{graphicx}
\usepackage{amsmath}

\title{Number Systems in Digital Electronics}
\author{Your}
\date{\today}

\begin{document}

% Title Slide
\begin{frame}
    \titlepage
\end{frame}

% Table of Contents
\begin{frame}{Outline}
    \tableofcontents
\end{frame}

% Definition of Number Systems
\section{Definition of Number Systems}
\begin{frame}{Definition and Origin of Number Systems}
    \begin{itemize}
        \item Number systems are methods of representing numerical values.
        \item They have evolved from ancient counting systems to modern digital representations.
        \item Essential in computing and digital circuit design.
    \end{itemize}
\end{frame}

% Types and Examples of Number Systems
\section{Types and Examples of Number Systems}
\begin{frame}{Types of Number Systems}
    \begin{itemize}
        \item Decimal (Base-10)
        \item Binary (Base-2)
        \item Octal (Base-8)
        \item Hexadecimal (Base-16)
        \item Binary Coded Decimal (BCD)
    \end{itemize}
\end{frame}

% Discussion on Number Systems
\section{Discussion on Number Systems}
\begin{frame}{Different Number Systems}
    \begin{itemize}
        \item Decimal: Most commonly used in daily life.
        \item Binary: Used in digital electronics and computing.
        \item Octal: Sometimes used in digital systems for compact representation.
        \item Hexadecimal: Used for memory addressing and color coding.
        \item BCD: Represents decimal digits in binary format.
    \end{itemize}
\end{frame}

% Tables of Number Systems
\section{Tables of Number Systems}
\begin{frame}{Number System Representations}
    \begin{table}[]
        \centering
        \begin{tabular}{|c|c|c|c|c|}
            \hline
            Decimal & Binary & Octal & Hexadecimal & BCD \\
            \hline
            0 & 0000 & 0 & 0 & 0000 \\
            1 & 0001 & 1 & 1 & 0001 \\
            2 & 0010 & 2 & 2 & 0010 \\
            3 & 0011 & 3 & 3 & 0011 \\
            4 & 0100 & 4 & 4 & 0100 \\
            5 & 0101 & 5 & 5 & 0101 \\
            \hline
        \end{tabular}
    \end{table}
\end{frame}

% Conversion of Number Systems
\section{Conversion of Number Systems}
\begin{frame}{Number System Conversions}
    \begin{itemize}
        \item \textbf{Binary to Decimal}: Sum the place values.
        \item \textbf{Hexadecimal to Decimal}: Multiply each digit by powers of 16.
        \item \textbf{Octal to Binary}: Convert each octal digit to a 3-bit binary.
        \item \textbf{BCD to Decimal}: Convert each 4-bit group to its decimal equivalent.
    \end{itemize}
\end{frame}

% Examples of Conversions with Floating Points
\begin{frame}{Examples of Conversions with Floating Points}
    \begin{itemize}
        \item Binary to Decimal: $101.101_2 = (1 \times 2^2) + (0 \times 2^1) + (1 \times 2^0) + (1 \times 2^{-1}) + (0 \times 2^{-2}) + (1 \times 2^{-3}) = 5.625_{10}$
        \item Decimal to Binary: $10.625_{10} = 1010.101_2$
        \item Hexadecimal to Decimal: $A.3_{16} = (10 \times 16^0) + (3 \times 16^{-1}) = 10.1875_{10}$
        \item Octal to Binary: $7.4_8 = 111.100_2$
    \end{itemize}
\end{frame}

% Operations on Binary Number System
\section{Operations on Binary Numbers}
\begin{frame}{Binary Arithmetic Operations}
    \begin{itemize}
        \item \textbf{Addition}: Follows rules of carrying.
        \item \textbf{Subtraction}: Uses borrowing like decimal subtraction.
        \item \textbf{Multiplication}: Follows the same pattern as decimal multiplication.
        \item \textbf{Division}: Similar to long division in decimal.
    \end{itemize}
\end{frame}

% Examples of Binary Arithmetic Operations
\begin{frame}{Examples of Binary Arithmetic Operations}
    \begin{itemize}
        \item \textbf{Addition:} $101_2 + 110_2 = 1011_2$
        \item \textbf{Subtraction:} $1101_2 - 101_2 = 1000_2$
        \item \textbf{Multiplication:} $101_2 \times 11_2 = 1111_2$
        \item \textbf{Division:} $1010_2 \div 10_2 = 101_2$
    \end{itemize}
\end{frame}

% Applications of Number Systems
\section{Applications of Number Systems}
\begin{frame}{Applications in Digital Electronics}
    \begin{itemize}
        \item Binary: Used in logic circuits and programming.
        \item Hexadecimal: Used in memory addressing and debugging.
        \item BCD: Used in digital clocks and calculators.
        \item Octal: Used in microprocessor and system design.
    \end{itemize}
\end{frame}

% Conclusion
\begin{frame}{Conclusion}
    \begin{itemize}
        \item Number systems are fundamental to digital electronics.
        \item Understanding conversions and operations is crucial.
        \item Applications span across computing, networking, and digital circuit design.
    \end{itemize}
\end{frame}

% Q&A Slide
\begin{frame}{Q&A}
    \centering
    Questions?
\end{frame}

\end{document}
