\documentclass[12pt]{article}
\usepackage{times}
\usepackage{setspace}
\usepackage{natbib}
\onehalfspacing

\begin{document}

The 2018 revelation that Facebook permitted Cambridge Analytica to access millions of users’ personal data without consent highlighted significant ethical concerns regarding data privacy and corporate responsibility. Digital forensics played a crucial role in investigating this breach to assess its scope and impact \cite{cadwalladr2018facebook}.

\section{Sensitive and Personal Data Collected During a Digital Forensics Investigation}
Investigators may collect the following:
\begin{itemize}
    \item Browsing History: websites visited
    \item Social Media Activity: Posts, likes, shares, friend lists.
    \item Multimedia Files: photos, videos, audio recordings
    \item Personal Identifiable Information (PII): Names, addresses, phone numbers.
    \item Financial Information: Latest transactions
    \item Authentication Credentials: Usernames, passwords
    \item Location Data
\end{itemize}

\section{Uganda’s Data Protection and Privacy Act, 2019}
Uganda’s Data Protection and Privacy Act, 2019 \cite{nsobani2019critical}, addresses issues related to the unauthorized access and misuse of personal data. In a scenario like the Cambridge Analytica case, the Act grants individuals several rights, including:
\begin{itemize}
    \item \textbf{Right to Be Informed}: Individuals have the right to know when their personal data is being collected, the purpose of its collection, and how it will be used \cite{ugandadataact2019}.
    \item \textbf{Right of Access}: Individuals can request access to their personal data held by an organization.
    \item \textbf{Right to Restrict Processing}: Individuals can request that their data be used only for certain purposes.
    \item \textbf{Right to Erasure (Right to Be Forgotten)}: Individuals can request the deletion of their personal data under certain circumstances.
    \item \textbf{Right to Data Portability}: Individuals can obtain and reuse their personal data across different services.
    \item \textbf{Right to Object}: Individuals can object to the processing of their personal data for specific purposes, such as direct marketing.
    \item \textbf{Rights Related to Automated Decision-Making and Profiling}: Individuals are protected against decisions made solely based on automated processing.
\end{itemize}

\section{The Role of the General Data Protection Regulation (GDPR) in Addressing Data Privacy Issues}
The GDPR, implemented by the European Union in 2018, is a comprehensive data protection regulation that sets a high standard for personal data handling. Its key provisions include \cite{gdpr2018}:
\begin{itemize}
    \item \textbf{Lawful Basis for Processing}: Organizations must have a valid reason to process personal data, such as user consent or legitimate interest.
    \item \textbf{Data Minimization}: Only data necessary for a specific purpose should be collected and processed.
    \item \textbf{Accountability and Governance}: Organizations are responsible for complying with GDPR principles and must demonstrate compliance.
    \item \textbf{Data Protection by Design and Default}: Data protection measures should be integrated into business processes from the outset.
    \item \textbf{Breach Notification}: Organizations must notify authorities and affected individuals of data breaches within 72 hours.
\end{itemize}

\section{Influence of GDPR \cite{regulation2016regulation} on the Cambridge Analytica Case}
Had GDPR been in effect during the Cambridge Analytica incident, it could have significantly emphasized the following:
\begin{itemize}
    \item \textbf{Consent Requirements}: Facebook would have been required to obtain explicit consent from users before sharing their data with third parties like Cambridge Analytica.
    \item \textbf{Heavy Penalties}: Non-compliance with GDPR can result in fines of up to 20 million euros or 4\% of the company’s global turnover \cite{gdprpenalties}.
    \item \textbf{Enhanced Individual Rights}: Users would have had the right to access, rectify, or erase their data, limiting Cambridge Analytica’s ability to misuse personal information.
    \item \textbf{Obligation to Report Breaches}: Facebook would have been obligated to promptly report the data breach to authorities and affected users, ensuring greater transparency.
\end{itemize}

\bibliographystyle{apalike}
\bibliography{ref}


\end{document}
